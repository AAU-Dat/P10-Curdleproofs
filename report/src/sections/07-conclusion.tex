

\section{Conclusion}\label{sec:conclusion}
After looking into Ethereum and implementing the de-anonymization attack on validators, we have found that the attack is possible.
With this reaffirms the findings of the~\gls{de-anon paper} and further highlights the security issues that the Ethereum network faces.
The issue being that even with for the low cost of 0 ether staked, an attacker can gain access lots of information about the validators in the network, most importantly their IP addresses.
This information can then be used to perform attacks on the network, such as a~\gls{dos} attack on block proposers, and potentially lead to a loss of money for the honest actors within the Ethereum network.


Our research has also shown that there is a difference in the makeup of the Holesky testnet compared to the mainnet.
The results of the de-anonymization attack on the Holesky testnet show that there are more irregular validators on the testnet compared to the mainnet.
It also shows that the amount of validators on an individual node is many times higher on the testnet compared to what you would find on the mainnet.
So even though the two networks are fundamentally the same, the ecosystem within them is different.