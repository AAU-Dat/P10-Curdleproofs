\section{Related Work}\label{sec:related-work}




\subsection{Single Secret Leader Election}\label{sec:related-work-SSLE}




\subsection{Shuffling algorithm}\label{sec:related-work-Shuffling-algorithm}
The shuffling algorithm used in curdleproofs has gone though many iterations and improvements in order to increase speed and reduce the size the proof.
This is because the proposer has a limited amount of time to propose a block in each slot, and the addition of the proof to the protocol increases the size of the block the proposers have to create.
This is the reason why the current implementation of curdleproofs has chosen the shuffling algorithm~\cite{cryptoeprint:2022/560} proposed by Larsen et al.

The way the shuffle works is by selecting 2 days' worth of proposers, and then shuffling the proposers over one day's worth of slots to create a new list of proposers for the following day.
In each slot a subset of the proposers are shuffled, and the rest are left unchanged.

Though experiments Larsen et al. has shown that after enough shuffles becomes secrue even in adversarial environments.
They also surgests that their may be room to lower the size of the subsets chosen in each lot without losing the security of the shuffle.
Thereby increasing the speed of the shuffle and reducing the size of the proof being added to the blockchain.

\subsection{Bulletproofs}\label{sec:related-work-bulletproofs}
A big inspiration for the curdleproofs protocol is the use of bulletproofs~\cite{bunz2018bulletproofs}.
Bulletproofs is a type of range proof that uses inner product arguments to prove that a committed value is within a certain range without revealing the value itself.
Bulletproofs is in itself not a zero-knowledge proof system, but with the help of Fiat Shamir~\cite{bunz2018bulletproofs} it can be used to create a zero-knowledge proof.
Bulletproofs also has had a few iterations and improvements to increase the speed and reduce the size of the proof since it was used in curdleproofs.
One of these is Bulletproofs+~\cite{chung2022bulletproofs+} which is a new version of bulletproof that uses a weighted inner product argument instead of the standard inner product argument to achieve a better performance.
Bulletproofs+ is also different because it is zero-knowledge proof by itself unlike the original bulletproofs.
A third version of the bulletproofs is Bulletproofs++~\cite{eagen2024bulletproofs++} which is a even newer version of bulletproofs that uses a new type of argument called the norm argument to achieve a better performance.
Unlike the two other proofs Bulletproofs++ is a binary range proof, which means that even if it is the fastest proof it is not suitable for the curdleproofs protocol due to the binary nature of the bulletproofs++.