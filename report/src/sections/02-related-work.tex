\section{Related Work}\label{sec:related-work}




\subsection{Single Secret Leader Election}\label{sec:related-work-SSLE}
A~\gls{ssle} is a protocol where a group of participants randomly elects only one leader from the group.
The identity of the leader is kept secret from all other participants so only the leader themselves know that they have been chosen.
The elected leader can then later publicly prove that they have been elected~\cite{10.1145/3419614.3423258}.
One of the use cases of~\gls{ssle} is to make~\gls{pos} cryptocurrencies more secure due to the added privacy that the proposer has.

One~\gls{pos} cryptocurrency that uses an~\gls{ssle} is Polkadot which uses Safrole as their~\gls{ssle} protocol~\cite{safrole}.



\subsection{Shuffling algorithms}\label{subsec:related-work-shuffling-algorithm}

The Håstad square shuffle~\cite{haastad2006square} is one of the proposed ways of integrating an~\gls{ssle}.
The Håstad square shuffle is a shuffling algorithm that shuffles a $n$ long vector with a shuffle size of $\sqrt {n}$.
The algorithm works by splitting the vector into $\sqrt {n}$ times $\sqrt {n}$ square matrix and for each step in the algorithm it switches between shuffling a row and a column.
The Håstad shuffle is more rigid than the shuffling algorithm used in curdleproofs~\cite{cryptoeprint:2022/560} because of the fixed size of the shuffle being $\sqrt {n}$.

The Feistel shuffle~\cite{Feistle} is the previous shuffle method used in the Whisk protocol~\cite{Whisk2024}.
The Feistel shuffle is a shuffling algorithm that works by taking $n$ number of trackers and arranging them in a $k$ times $k$ matrix.
Each round the $i$-th proposer selects the $i$-th row of the created matrix and shuffles it in the form $F(x,y)=(y,x+y^3 mod k)$.
The Feistel shuffle was then later replaced by the shuffle proposed be Larsen et al.~\cite{cryptoeprint:2022/560} because of the Feistel shuffle being too slow to shuffle the list of proposers.


\subsection{Bulletproofs}\label{subsec:related-work-bulletproofs}
A big inspiration for the Curdleproofs protocol is bulletproofs~\cite{bunz2018bulletproofs}.
Bulletproofs is a type of range proof that uses inner product arguments to prove that a committed value is within a certain range without revealing the value itself.
Bulletproofs is in itself not a zero-knowledge proof system, but with the help of Fiat Shamir~\cite{bunz2018bulletproofs} it can be used to create a zero-knowledge proof.
Bulletproofs also has had a few iterations and improvements to increase the speed and reduce the size of the proof since it was used in curdleproofs.
One of these is Bulletproofs+~\cite{chung2022bulletproofs+} which uses a weighted inner product argument instead of the standard inner product argument to achieve a better performance.
Bulletproofs+ is also a zero-knowledge proof by itself unlike the original bulletproofs.
A third version of the Bulletproofs protocol is Bulletproofs++~\cite{eagen2024bulletproofs++} which uses a new type of argument called the norm argument to achieve a better performance.
Unlike the two other proofs Bulletproofs++ is a binary range proof, which means that even if it is the fastest proof it is not suitable for the Curdleproofs protocol due to the binary nature of the bulletproofs++.