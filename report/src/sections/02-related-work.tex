\section{Related Work}\label{sec:related-work}




\subsection{Whisk}\label{sec:related-work-whisk}
Ethereum currently has an improvement proposal suggesting the implementation of called Whisk~\cite{Whisk2024}.
Whisk is a zero-knowledge proof system that allows for the verification of the correctness of a shuffle without revealing the input or output.
It is based on the concept of inner product arguments and uses elliptic curve cryptography to achieve its goals.
Whisk is designed to be efficient and scalable, making it suitable for use in Ethereum and other blockchain systems.
Whisk is one of the suggested solutions to attacks on the Ethereum network targeting block proposers.
With the help of the zero-knowledge proofs Whisk help make the previous public proposer list and order into a system where only the proposer can see if it is their turn to propose a block, and they can proof that to be the case.


\subsection{Shuffling algorithm}\label{sec:related-work-Shuffling-algorithm}
The shuffling algorithm has gone though many iterations and improvements in order ti increase speed and reduce the size the proof.
This is because the proposer has a limited amount of time to propose a block in each slot, and the addition of the proof to the protocol adds increase the size of the block.
This is the reason why the current implementation of curdleproof has chosen the shuffling algorithm~\cite{cryptoeprint:2022/560} be Larsen et al.

