

\section{Future work}\label{sec:future-works}

\subsection{Potential mitigations}\label{subsec:potential-mitigations}
At the moment there exists an improvement proposal
to include a~\gls{ssle} mechanism, called Whisk, in Ethereum~\cite{EthereumResearchSSLE2024}.
This method aims to improve the security of the network by obfuscating the identity of the proposer.
It does so by making the validators commit to a shared secret which can be bound to a validators identity and randomized to make to match the specific validator's identity.
Every epoch a random set of validators are chosen to gather commitments from a set of validators using~\gls{randao}.
Proposers then shuffle the commitments over the durration of 8182 slots.
Then~\gls{randao} is used to map the shuffled list onto the slots.
Validators can now decrypt the commitment that matches their identity and propose a block in the slot that they are assigned to.
This whole process is and example of~\gls{zkp} where the point is that every validator is able to prove that it is their turn to propose a block without revealing their identity.
This would make it harder
for an adversary to perform the Proposer~\gls{dos} attack since the adversary would not know which validator to target.
But even if this stops the~\gls{dos} attack, it does not prevent the de-anonymization of the validators.
But hindering a~\gls{dos} attack in itself is a good reason
to look at implementing~\gls{ssle} in Ethereum as a further step to improving the security of the network.
