
\begin{abstract}
    Ethereum prioritizes keeping the blockchain secure, and this will continue due to the large amount of money invested in the network.
    One of the network's most central aspects is the validators, who are responsible for producing the blocks that make up the chain.
    However, security and privacy are at risk as Heimbach et al. explore a vulnerability, enabling de-anonymizing validators and linking IP addresses to them.
    This paper reproduces the de-anonymization on the Ethereum Holesky testnet, verifying the continued existence of the vulnerability.
    The reproduction shows similar results, de-anonymizing close to the same portion of validators found through the experiment.
    Differences from de-anonymization on a testnet contrary to the mainnet are discussed, showing, among other things, that nodes generally run more validators on the testnet.
    The consequences of the validator de-anonymization are also explored, and a potential denial-of-service attack on block proposers is described.
    The Denial-of-Service attack allows an adversary to halt de-anonymized block proposers, resulting in them missing their proposal opportunity and getting penalized.
\end{abstract}

\begin{IEEEkeywords}
    Ethereum, Blockchain, Distributed Systems, Privacy Attack
\end{IEEEkeywords}
