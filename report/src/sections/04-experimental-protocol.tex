

\section{Experimental protocol}\label{sec:experimental-protocol}
\subsection{Attack implementation}\label{subsec:attack-implementation}
As mentioned in x~\todo{reference to subsection above},
the work in this paper revolves around modifying a fork of the Prysm consensus client.
An important thing to keep in mind is that the core functionality of Prysm is never changed.
So no changes have been made to variables already being used in the Prysm client.

Instead, to be able to implement the attack,
it is a case of finding where the different functionalities are written in the code.
It is enough to modify just two classes in the code to reach a point where the attack is possible.

As mentioned in~\todo{reference to subsection above} two entities are logged, peers and attestations.
This translates almost directly to the code in the Prysm client, as each of these can be handled by a class each. \todo{How much are we allowed to say? We assume mentioning which classes/functions to modify is unethical}

\subsubsection{Peers}\label{subsubsec:peers}
The logs of the attestations share similarities with the logs of the peers, so the peers will be described first.

A pointer to a \texttt{Service} object is typically maintained in the different classes of the Prysm code.
This object is an implementation of an interface responsible for handling functionality coupled with the given class.
The class that handles~\gls{p2p} connections has got this as well, with several specific functions for handling a peer.

Because of said object, information about all earlier or connected peers is available.
Set to run every minute, the modified code requests all peers and iterates through them,
logging the information mentioned in~\todo{reference above subsection}.


Also in each iteration,
the collected information is sent to an \texttt{sqlite3} database such the data can be processed later.

\subsubsection{Attestations}\label{subsubsec:attestations}
Everything logged in for the peers is also logged for the attestations,
as there is always an underlying node sending each attestation.

The class handling the attestations also has a \texttt{Service} interface implementation.
This implementation primarily includes functions specific for the attestations, and not peers.
Therefore, each \texttt{Service} object has a pointer to a configuration of the running chain.
Referencing this pointer allows the class to access functions as the ones available in the~\gls{p2p} class.

Instead of being logged for once every minute, as in~\autoref{subsubsec:peers},
we log information about an attestation whenever we receive one.

With the proposer DoS attack in mind, we are interested in receiving the public key of the validator,
who is the creator of the attestation.
Getting this requires little effort,
as there already exists a function which uses the attestation's data to get public keys.
We can use this public key in another existing function to get the ID of the validator.

After collection all the information available for an attestation,
the data is logged to a database as in~\autoref{subsubsec:peers}.