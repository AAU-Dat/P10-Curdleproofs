

\section{Experimental protocol}\label{sec:experimental-protocol}
In this section, we describe the setup of the experiment that will be conducted as well as the experiment itself.
The section will also cover the workings of the attack that will ve performed.

\subsection{Setup}\label{subsec:setup}
The setup of the experiment is as follows: First we modified a prysm client to include logging of attestations received by the client and all the peer that the node has discovered.
The client was then run on a virtual machine hosted on Strato Claaudia through the Aalborg University.

To start receiving attestations and reaching and connecting other peers, the client was connected to the Holesky testnet.
Since the experiment is of an adversarial nature, the client was connected to a testnet instead of the mainnet.

While the client was running, it was able to log information from the received attestations as well as on the peers it discovered.
Among other things, the client logged peer~\glspl{enr}, the subnets that the attestations were sent to, the subnet that the peer was in and the public key associated with the attestation.
\todo {mabye add image of tableheaders here for vizualization}
The data that was logged was then stored in a database for further analysis.
After this, the peers were sorted into the same four categories as mentioned in the~\gls{de-anon paper}~\cite{heimbach2024deanonymizingethereumvalidatorsp2p}.
The categories are:
\begin{itemize}
    \item Deanonymized
    \item No validators
    \item 64 subnets
    \item Rest
\end{itemize}

\textbf{Deanonymized:} One or more validators have been discovered on this peer.

\textbf{No validators:} No validators have been discovered on this node, as we did not receive a non-backbone attestation.

\textbf{64 subnets:} The peer was subscribed to all 64 subnets, so we are not able to tell backbone attestations from non-backbone attestations.

\textbf{Rest:} Peers where we receive non-backbone attestations, but we were not able to identify any validators.

When it comes to deanonymizing the validators, we used the same heuristic as in the~\gls{de-anon paper}~\cite{heimbach2024deanonymizingethereumvalidatorsp2p}.
The heuristic includes four criteria being:
\begin{itemize}
    \item The proportion of non-backbone attestations for validator v exceed
    \begin{equation}
                                                                                 0.9*\left(\frac{64-n_{sub}(p)}{64}\right)
                                                                                 \label{eq:heurestic}
    \end{equation} where nsub(p) is the average number of subnets the peer is subscribed over the connection’s duration.
    \item The peer is not subscribed to all 64 subnets.
    \item TWe receive at least every tenth attestation we expect for the validator v from the peer.
    \item The number of attestations we receive for the validator v from the peer p exceeds the mean number of attestations per validator from peer p by two standard deviations.
\end{itemize}
\todo{is it fine to just copy the criteria from the paper word for word?}

\subsection{Attack implementation}\label{subsec:attack-implementation}
As mentioned in~\autoref{subsec:setup},
the work in this paper revolves around modifying a fork of the Prysm consensus client.
An important thing to keep in mind is that the core functionality of Prysm is never changed.
So no changes have been made to variables already being used in the Prysm client.

Instead, to be able to implement the attack,
it is a case of finding where the different functionalities are written in the code.
It is enough to modify just two classes in the code to reach a point where the attack is possible.

As mentioned in~\autoref{subsec:setup} two entities are logged, peers and attestations.
This translates almost directly to the code in the Prysm client, as each of these can be handled by a class each.

\subsubsection{Peers}\label{subsubsec:peers}
The logs of the attestations share similarities with the logs of the peers, so the peers will be described first.

A pointer to a \texttt{Service} object is typically maintained in the different classes of the Prysm code.
This object is an implementation of an interface responsible for handling functionality coupled with the given class.
The class that handles~\gls{p2p} connections has got this as well, with several specific functions for handling a peer.

Because of said object, information about all earlier or connected peers is available.
Set to run every minute, the modified code requests all peers and iterates through them,
logging the information mentioned in~\autoref{subsec:setup}.


Also in each iteration,
the collected information is sent to an \texttt{sqlite3} database such the data can be processed later.

\subsubsection{Attestations}\label{subsubsec:attestations}
Everything logged in for the peers is also logged for the attestations,
as there is always an underlying node sending each attestation.

The class handling the attestations also has a \texttt{Service} interface implementation.
This implementation primarily includes functions specific for the attestations, and not peers.
Therefore, each \texttt{Service} object has a pointer to a configuration of the running chain.
Referencing this pointer allows the class to access functions as the ones available in the~\gls{p2p} class.

Instead of being logged for once every minute, as in~\autoref{subsubsec:peers},
we log information about an attestation whenever we receive one.

With the proposer DoS attack in mind, we are interested in receiving the public key of the validator,
who is the creator of the attestation.
Getting this requires little effort,
as there already exists a function which uses the attestation's data to get public keys.
We can use this public key in another existing function to get the ID of the validator.

After collection all the information available for an attestation,
the data is logged to a database as in~\autoref{subsubsec:peers}.

\subsubsection{Exploiting RANDAO for a Proposer DoS attack}
Applying the aforementioned heuristics to the logged data will enable us to get a list of de-anonymized validators.
Having de-anonymized validators means that the~\gls{randao} proposer selection is vulnerable, but not limited, to a DoS attack.
More specifically, it is possible to perform the Proposer DoS attack as mentioned in~\autoref{subsec:proposer-dos-attack}.

In~\autoref{subsec:randao}, we found that the proposers for an entire epoch are chosen at least one epoch in advance.
Gathering the proposer duties can be easily obtained, as Ethereum provides an API endpoint for the blockchain, providing information on the chain.~\footnote{The API calls are found \href{https://ethereum.github.io/beacon-APIs/\#/Validator/getProposerDuties}{here.}}.
The adversary would only need an Ethereum node to use this API.
A validator is not required.
This is especially beneficial, as one can perform the attack without any financial risks.

The output of the call to the API returns the chosen validator's public key, ID, and the slot in which they are chosen to propose a block.

Therefore, a list of the de-anonymized validators could be iterated, where the goal of the adversary would be to match the proposers for the upcoming epoch.
Included in the de-anonymization is the IP-addresses of the validator as well as their ID and public key.
So having matched the upcoming proposers with de-anonymized validators, an adversary now ideally knows the IP-addresses of several upcoming proposers.

This allows for the adversary to DoS the specific IP-address, which could leave the proposer inactive at their designated slot.

\subsubsection{Reasons for doing the attack}

Several reasons can explain why an adversary would be interested in performing this kind of attack.
This counts both for adversaries running a node with validators and adversaries only running a node in the network


The first reason which only requires de-anonymization, could be to exploit the inactivity penalization, which is mentioned in~\autoref{subsubsec:valpunish}.
The adversary could make validators lose money for being inactive.
For the adversaries themselves, they don't receive any monetary rewards from this attack. 
One could be interested in getting people kicked from being a validator for being 16 ether.
But it can quickly be derived from~\autoref{eq:burn} that it would require the DoS attack to be ongoing for at least 9645 epochs before a validator goes from 32 ether to under 16 ether.


Having more focus on targeting upcoming proposers, the adversary can limit the rewards that the proposers are promised.
Proposers get the best reward after proposing a block, see~\autoref{subsubsec:valrewards}, so the adversary could limit the rewards given out to proposers.
In the worst case, this could lead to a decrease in the total amount of ether, as it is dynamically changing with the chain.


Both of the aforementioned reasons only require a node in Ethereum, but the adversary themselves do not gain any reward from the attack.
This could be a possibility if the adversary is in control of a validator.
The reason for this is found in the~\gls{mev}~\cite{mev}.

~\gls{mev} is a reference to the maximum value one can extract from a block production opportunity.
It is a reward that the proposer can get besides rewards gotten from working on the blockchain, as mentioned in~\autoref{subsubsec:valrewards}.
The idea of~\gls{mev} is that \textit{searchers} search the network for profitable transaction opportunities.
These are, for instance, opportunities where people are willing to pay a high transaction fee to the proposer for inclusion in the block.
These transactions are then added to the pool of transactions to be added in the chain.
There are more implementations of these searchers, fx Flashbots and MEV-Boost.
It has been estimated that ~90\% of all Ethereum validators use this kind of software~\cite{cryptoeprint:2023/760}.

So, let us assume a scenario, where an adversary has a validator that is set to propose the block at slot $n$.
It could then be beneficial for the adversary to perform a~\gls{dos} attack on the proposers of slot $n-1, n-2$, and so on.
Beginning from slot $n-1$ and down, the more consecutive proposers that are successfully attacked, the more potential gain there is for the adversary.
This is because the high-paying transactions that were to be included in the previous slots now can be included in the adversary's slot.
Therefore, the adversary potentially gains a greater reward from including those transactions with higher fees, that are now available because of the attack.