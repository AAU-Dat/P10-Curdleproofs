
\section{Experimental Protocol}\label{sec:experimental-protocol}
In this section, we will describe how our experiments are run, and what we want to measure.
We also discuss which parameters we can tweak in the different experiments that we have.

The experiments are run on a virtual machine hosted on Strato CLAAUDIA at Aalborg University.
The machine is using an Intel Xeon Cascadelake processor, CPU family 6, model 85.
It has 16 virtual CPUs and 64 GB of RAM available.
The virtual machine is running Ubuntu Server 24.


\subsection{CAAUrdleproof}\label{sec:CAAUrdleproof-experiment}
In this experiment we measure the time to run the CAAUrdleproofs protocol.
The results will be compared to those of Curdleproofs, which we re-run on our own hardware.
As Curdleproofs already has a Rust benchmark implemented, we will be using that same benchmark for both protocols.
The parameter that we want to change between benchmark runs is the shuffle size, $\ell$.

In CAAUrdleproofs, we will test the protocol with $\ell=\{8,9,\dots,256\}$.

Since Curdleproofs is unable to run benchmarks, unless the shuffle size is a power of two, those benchmarks will be run on values $\ell=\{8,16,32,64,128,256\}$.




\subsection{Shuffle security}\label{subsec:experimental-protocol-shuffle-security}
In this experiment we run the shuffle protocol with varying shuffle sizes and varying number of adversarial tracked ciphertexts.
The purpose of this experiment is to find the lowest possible shuffle size that is still secure against adversarial tracking.
We therefore run the experiment with shuffle sizes, $\ell$, between 64 and 128.
For the number of adversarial tracked ciphertexts, we use the values $\alpha=\{1/2,1/3,1/4\}$

Because Curdleproofs is meant to be used in an Ethereum setting, all the experiments were done with a maximum of 8192 shuffles.
Also, the experiments shuffle over a set of 16,384 ciphertexts.
Both of these numbers come from the Ethereum Whisk proposal~\cite{Whisk2024}.


Every experiment is run 1000 times to avoid statistical uncertainty.
As done by the shuffle authors~\cite{cryptoeprint:2022/560}, we will denote the 20th, 40th, 60th, 80th, and 100th percentile on when the shuffle is deemed secure by the experimental runs.

