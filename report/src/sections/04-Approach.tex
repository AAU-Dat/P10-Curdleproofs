\section{approach}\label{sec:approach}


\subsection{Shuffle security}\label{sec:approach-shuffle-security}
The shuffle method proposed by Larsen et al.~\cite{cryptoeprint:2022/560} that was used in curdleproofs is based on the idea of shuffling a list of proposers over a set of slots.
The shuffle itself however is not too complex.
A formal definition of the shuffle is given in~\autoref{fig:shuffle}.

\begin{figure}[ht]\label{fig:shuffle}

    \begin{framed}
        \[
            \Pi(c_1, \ldots, c_n)
        \]
        \rule{\linewidth}{0.4pt}

        \noindent
        \textbf{for} $t \in [T]$ \textbf{:}
        \begin{itemize}
            \item[$S_t$] picks random $\{i_1, \ldots, i_k\} \subset [n]$
            \item[$S_t$] computes $(\tilde{c}_{i_1}, \ldots, \tilde{c}_{i_k}) \leftarrow \text{Shuffle}(c_{i_1}, \ldots, c_{i_k})$
            \item[$S_t$] publishes $(\tilde{c}_{i_1}, \ldots, \tilde{c}_{i_k})$
        \end{itemize}
    \end{framed}
    \caption{Distributed shuffling protocol.}
\end{figure}

Here the set $(c_1, \ldots, c_n)$ is a set of ciphertexts that are shuffled over $T$ slots.

\subsection{Springproofs}\label{sec:approach-springproofs}



\subsection{implementation}\label{sec:approach-implementation}


