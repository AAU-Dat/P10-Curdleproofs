%! Author = Anders
%! Date = 15-10-2024

\section{Related Work}\label{sec:related-work}
The usage of~\glspl{zkp} in Ethereum is not a new concept.
In fact, it currently uses them both on- and off-chain.
The following provides a short overview of some of the already existing solutions as well as one still being in development.
\subsection{MACI}\label{subsec:MACI}
In defense of a potential bribery attack, see~\autoref{sec:background}, the Ethereum blockchain implements a private voting system called \gls{maci}~\cite{EthereumAttacks2024,MACI2022}.

What ~\gls{maci} does is essentially hiding what each person has voted for.
It does so by demanding the voters to send their votes encrypted to a central coordinator.
This coordinator constructs~\gls{zk-snark} proofs, which verifies that all messages were processed correctly, and that the final result corresponds to the sum of all valid votes.

As votes are now hidden, the adversary is not able, by oneself, to prove that the bribee voted in way of said bribery.
Though the bribee could decrypt their own message and show the vote to the adversary.

~\gls{maci} has fixed this problem by implementing public key switching.
This means that a voter can request a new public key.
In addition to this, a vote is only valid if it uses the most recent public key of the voter.
Therefore, a bribee can show its first vote obeying the adversary, generate a new public key, and send a new, now honest, vote.
The old vote will then become invalid as it uses a deprecated public key.

\subsection{Off-chain roll-ups}\label{subsec:off-chain-roll-ups}
\subsection{Whisk}\label{subsec:whisk}

