\section{Related Work}\label{sec:related-work}




\subsection{Single Secret Leader Election}\label{sec:related-work-SSLE}
A~\gls{ssle} is a protocol where a group of participants randomly elects only one leader from the group.
The identity of the leader is kept secret from all other participants so only the leader themselves know that they have been chosen.
The elected leader can then later publicly prove that they have been elected~\cite{10.1145/3419614.3423258}.

Leading research on~\gls{ssle} includes proposals for post-quantum secure protocols based on Learning With Errors and Ring Learning With Errors~\cite{cryptoeprint:2023/1241}.
This work also constructs a new concept called~\gls{rrc} for easier work with such protocols.
\gls{rrc} is based on the commit-and-shuffle approach also used in Whisk.

One of the use cases of~\gls{ssle} is to make~\gls{pos} blockchains more secure due to the added privacy that the proposer has.

One~\gls{pos} blockchain that uses an~\gls{ssle} is Polkadot which uses Safrole as their~\gls{ssle} protocol~\cite{safrole}.
Safrole is the production version of the research protocol Sassafras~\cite{sassafras}.
In this, validators each produce a number of tickets, some of which are winning, depending on some threshold.
A~\gls{zk-snark} is then used to prove that a ticket is winning, after which the winning tickets are published to the chain.
A randomization algorithm will then pick, from all the winning tickets, proposers for all the slots two epochs later.



\subsection{Shuffling algorithms}\label{subsec:related-work-shuffling-algorithm}

The Håstad square shuffle~\cite{haastad2006square} is one of the proposed ways of shuffling, which could be integrated in a shuffling~\gls{ssle} such as Whisk.
The Håstad square shuffle is a shuffling algorithm that shuffles a vector with $n$ items with a shuffle size of $\sqrt {n}$.
The algorithm works by re-arranging the vector into a~$\sqrt{n}\times\sqrt{n}$ square matrix.
It then works in time steps, starting at 1.
For each odd step, each column and its elements are shuffled independently.
For each even step, each row and its elements are shuffled independently as well.
Håstad shows that at least three time steps are needed for the shuffle to be secure.
The Håstad shuffle is more rigid than the shuffling algorithm used in curdleproofs~\cite{cryptoeprint:2022/560} because of the fixed size of the shuffle being $\sqrt{n}$.

The Feistel shuffle~\cite{Feistle} is a previously used shuffle method in the Whisk protocol~\cite{Whisk2024}.
It takes $n$ number of validator trackers and arranges them in a $k\times k$ matrix.
Each round the $i$-th proposer selects the $i$-th row of the created matrix and shuffles it in the form $F(x,y)=(y,x+y^3\text{ mod }k)$.
The Feistel shuffle was later replaced by the shuffle proposed by Larsen et al.~\cite{cryptoeprint:2022/560}.
Ethereum mentioned the reason for this to be that the shuffle by Larsen et al.\ provides a simpler protocol~\cite{Whisk2024}.

\subsection{Bulletproofs}\label{subsec:related-work-bulletproofs}
A big inspiration for the Curdleproofs protocol is bulletproofs~\cite{bunz2018bulletproofs}.
Bulletproofs is a type of range proof that uses inner product arguments to prove that a committed value is within a certain range without revealing the value itself.
Bulletproofs is in itself not a zero-knowledge proof system, but with the help of Fiat Shamir~\cite{bunz2018bulletproofs} it can be used to create a zero-knowledge proof.
Bulletproofs also has had a few iterations and improvements to increase the speed and reduce the size of the proof since it was used in curdleproofs.

One of these is Bulletproofs+~\cite{chung2022bulletproofs+} which uses a weighted inner product argument instead of the standard inner product argument to achieve a better performance.
Bulletproofs+ is also a zero-knowledge proof by itself unlike the original bulletproofs.
Trying to modify Curdleproofs with the weighted inner product argument introduces complications that would need larger modifications and is therefore not suitable.
This can be seen in~\autoref{sec:curdleproofs-weighted-inner-product-argument-modification-attempt}

A third version of the Bulletproofs protocol is Bulletproofs++~\cite{eagen2024bulletproofs++} which uses a new type of argument called the norm argument to achieve a better performance.
This comes from the prover only needing to commit to a single vector, rather than two.
Therefore, with the two vectors, $x$ and $y$ of a standard~\gls{ipa}, they need to assume $x=y$ for their protocol to work.
Then, along with the norm being weighted, which raises the same complications as with Bulletproofs+, this makes it unsuitable for Curdleproofs.