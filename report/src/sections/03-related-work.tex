%! Author = Anders
%! Date = 15-10-2024

\section{Related Work}\label{sec:related-work}
The usage of~\glspl{zkp} in Ethereum is not a new concept.
In fact, it currently uses them both on- and off-chain.
The following provides a short overview of some of the already existing solutions as well as one still being in development.

\subsection{Off-chain roll-ups}\label{subsec:off-chain-roll-ups}
As a result of Ethereum's popularity, the network could easily get congested if developers were not actively trying to distribute computations~\cite{EthereumScaling2024}.

One of the first ideas was to introduce an on-chain technique called sharding.
It is a technique where the database would be split into different parts between subsets of validators.
Sharding was never deployed on the blockchain though, and instead Ethereum uses off-chain solutions to off-load computations.
The idea of off-chain solutions, called L2-roll-ups, is that users commit their work to off-chain nodes.
These perform the computations, thereafter they submit the work to the Ethereum Mainnet chain.

One of these solutions is called a~\gls{zk-rollup}.

