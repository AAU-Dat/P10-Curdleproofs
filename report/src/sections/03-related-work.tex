

\section{Related Work}\label{sec:related-work}
\subsection{Tools}\label{subsec:tools}
%The usage of~\glspl{zkp} in Ethereum is not a new concept.
%In fact, it currently uses them both on- and off-chain.
%The following provides a short overview of some of the already existing solutions as well as one still being in development.

%\subsection{Off-chain roll-ups}\label{subsec:off-chain-roll-ups}
%As a result of Ethereum's popularity, the network could easily get congested if developers were not actively trying to distribute computations~\cite{EthereumScaling2024}.

%One of the first ideas was to introduce an on-chain technique called sharding.
%It is a technique where the database would be split into different parts between subsets of validators.
%Sharding was never deployed on the blockchain though, and instead Ethereum uses off-chain solutions to off-load computations.
%The idea of off-chain solutions, called L2-roll-ups, is that users commit their work to off-chain nodes.
%These perform the computations, thereafter they submit the work to the Ethereum Mainnet chain.
%
%One of these solutions is called a~\gls{zk-rollup}.

\subsection{DoS Attacks}\label{subsec:dos-attack}
Some of the instances of~\gls{dos} attacks that are seen on ethereum ranges from attacks on the proposers to attacks that seek to slow down the network itself.

\subsubsection{Underpriced opcodes}\label{subsubsec:underpriced-opcodes}
A known attack aims to slow down the network by using underpriced opcodes to create a block that is hard to process~\cite{10.1145/3391195,9815256}.

\subsubsection{Empty accounts in the state trie}\label{subsubsec:empty-accounts-in-the-state-trie}
Another way to slow down the network is to create empty accounts that are hard to process~\cite{empty-account-mitigation,empty-account-eip-mitigation}.
This attack, however, is outdated and has been mitigated by making it near impossible to create empty accounts in the network.

\subsubsection{Proposer DoS Attack}\label{subsubsec:proposer-dos-attack}
In this subsection, we will be describing the attack that we will be using as a basis for our experiment in ~\autoref{sec:experimental-protocol}.
The attack is a~\gls{dos} attack that aims at hitting the proposers selected for creating blocks in the chain.
Ethereum themselves have mentioned it as a potential attack, and with the current implementation of the consensus algorithm, it seems that this attack is possible to perform~\cite{EthereumSSLE2024,EthereumAttackDefense2024}.

It has been our interest to research the feasibility of this attack and the ones mentioned in~\autoref{sec:attacks-on-ethereum}.
This has proven to be a difficult task, given that most of our researched attacks happen in the consensus- or execution layer.
Therefore, as a result of the blockchain algorithm, we are not able to clarify the feasibility of the attacks that we have found.
For this reason, we have chosen the \textit{Proposer~\gls{dos} attack} as it seems exciting, has not been mitigated yet, and a potential solutions seems to include a~\gls{zkp}.
%\todo{Does it make sense to mention \# skipped blocks pr day even though we are unsure if any of these are attacks}

The attack possible is because the consensus mechanism uses a publicly known function for choosing the upcoming block proposers.
The adversary is therefore able to compute this in slight advance of the blockchain, s.t.\ each proposer is now known.
After this, the adversary can map the proposer's IP addresses and overload their connection.
A successful attack would leave a proposer unable to propose their block in time.\todo{Should possibly be explained in more detail}

