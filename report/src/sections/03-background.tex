
\section{Background}\label{sec:background}


\subsection{Notation}\label{sec:background-notation}
The notation used throughout this paper can be seen in~\autoref{tab:notation}.
\begin{table*}[ht]
    \centering
    \begin{tabular}{|l|l|}
        \hline
        \textbf{Symbol} & \textbf{Description} \\
        \hline
        $\mathbb{G}$ & Cyclic, additive, group of prime order $p$ \\
        \hline
        $\mathbb{Z}_p$ & Ring of integers modulo $p$ \\
        \hline
        $\mathbb{G}^n,\;\mathbb{Z}_p^n$ & Vector spaces of dimension $n$ over $\mathbb{G}$ and $\mathbb{Z}_p$ \\
        \hline
        $\mathbb{Z}_p^*$ & Multiplicative group $\mathbb{Z}_p\setminus\{0\}$ \\
        \hline
        $H\in\mathbb{G}$ & Generator of $\mathbb{G}$ \\
        \hline
        $\mathbf{\gamma}\in\mathbb{Z}_p^{\lceil\log n\rceil}$ & Uniformly distributed challenges \\
        \hline
        $\mathbf{a}\in\mathbb{F}^n$ & Vector $\mathbf{a}=(a_1,\dots,a_n)\in\mathbb{F}^n$ \\
        \hline
        $\mathbf{A}\in\mathbb{F}^{n\times m}$ & Matrix with $n$ rows and $m$ columns \\
        \hline
        $\mathbf{b}=c\cdot \mathbf{a}\in\mathbb{Z}_p^n$
        & The vector where $b_i = c\,a_i$, with scalar $c\in\mathbb{Z}_p$ and $\mathbf{a}\in\mathbb{Z}_p^n$ \\
        \hline
        $\mathbf{a}\times \mathbf{b}=\sum_{i=1}^n a_i\cdot b_i$
        & Inner product of $\mathbf{a},\mathbf{b}\in\mathbb{F}^n$ \\
        \hline
        $\mathbf{G}=(g_1,\dots,g_n)\in\mathbb{G}^n,\mathbf{G'}=(g'_1,\dots,g'_n)\in\mathbb{G}^n$
        & Vectors of generators (for Pedersen commitments) \\
        \hline
        $A=a\times G=\sum_{i=1}^n a_i\cdot G_i$
        & Binding (but not hiding) commitment to $a\in\mathbb{Z}_p^n\in $ \\
        \hline
        $\mathbf{r}_A\in\mathbb{Z}^n$ & Blinding factors, e.g.\ $A=\mathbf{a}\times\mathbf{G} + \mathbf{r}_A \times \mathbf{G}$ is a Pedersen commitment to $\mathbf{a}$ \\
        \hline
        $\mathbf{a}\parallel \mathbf{b}\in\mathbb{Z}_p^{n+m}$
        & Concatenation: if $\mathbf{a}\in\mathbb{Z}_p^n$, $\mathbf{b}\in\mathbb{Z}_p^m$, then $\mathbf{a}\parallel \mathbf{b}\in\mathbb{Z}_p^{n+m}$ \\
        \hline
        $\mathbf{a}_{[:k]}=(a_1,\dots,a_k)\in\mathbb{F}^k, \; \mathbf{a}_{[k:]}=(a_{k+1},\dots,a_n)\in\mathbb{F}^{n-k}$
        & Slices of vectors (Python notation) \\
        \hline
        \{\textit{Public Input}, \textit{Witness}\}: Relation
        & Relation using the specified public input and witness \\
        \hline
    \end{tabular}
    \caption{Notation used throughout the paper.}
    \label{tab:notation}
\end{table*}

\subsection{Security Assumptions}\label{subsec:security-assumptions}
Since this work is based on the existing Curdleproofs protocol~\cite{Curdleproofs}, it inherits the same security assumptions.
Our work therefore runs as a public coin protocol in any cryptographic group where~\gls{ddh} holds~\cite{10.1007/BFb0054851}.

\subsection{Zero-knowledge proofs}\label{sec:background-zkps}
Curdleproofs is a zero-knowledge proof system, which means that it allows a prover to convince a verifier that they know a secret without revealing the secret itself.
within the context of Ethereum it could be the ability to convince someone that a transaction is valid without revealing information about the transaction such as the value of it.


