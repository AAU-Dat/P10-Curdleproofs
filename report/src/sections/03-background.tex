
\section{Background}\label{sec:background}


\subsection{Notation}\label{sec:background-notation}


\subsection{Whisk}\label{sec:related-work-whisk}

\subsubsection{Ethereum}\label{sec:background-ethereum}
Ethereum is a decentralized blockchain platform that enables developers to build and deploy smart contracts and decentralized applications.
It is the second-largest blockchain platform by market capitalization and has a large and active developer community.
Ethereum uses a proof-of-stake consensus mechanism, which allows users to validate transactions and create new blocks by staking their Ether (ETH) tokens.
The Proof-of stake protocol works in epochs of 32 slots, where each slot is 12 seconds long.
In each slot a proposer is chosen to propose a block thereby allowing the network to reach consensus on the state of the blockchain.

\subsubsection{Proposer doss attack}\label{sec:background-proposer-doss-attacks}
An attack on the Ethereum network that was discovered by Heimbach et al.~\cite{heimbach2024deanonymizingethereumvalidatorsp2p} is the deanonymization attack.
In our preliminary work~\cite{ouroldpaper} we have shown that the attack still possible to perform on the Ethereum network and using that attack a proposer DoS can be preformed.
The proposer DoS attack is a type of attack that targets the block proposers making them unable to propose blocks.
An adversary can use the proposer DoS attack to prevent a proposer from receiving rewards from proposing a block and increase their rewards~\cite{EthereumSSLE2024}.

\subsubsection{Mitigation}\label{sec:background-mitigation}


%Ethereum currently has an improvement proposal suggesting the implementation of a protocol called Whisk~\cite{Whisk2024}.
%Whisk is a zero-knowledge Single secret leader election (SSLE) system that Through a zero-knowledge argument called curdleproofs~\cite{Curdleproofs} allows for the verification of the correctness of a shuffle without revealing the input or output.
%It is based on the concept of inner product arguments and does not require a honest setup.
%It uses elliptic curve cryptography based on the BLS12-381 curve to achieve its goals.
%
%Whisk is designed to be efficient and scalable, making it suitable for use in Ethereum and other blockchain systems.
%Whisk is one of the suggested solutions to attacks on the Ethereum network targeting block proposers.
%With the help of the zero-knowledge proofs, Whisk helps make the previous public proposer list and order into a system where only the proposer can see if it is their turn to propose a block, and they can proof that to be the case.


\subsection{Zero-knowledge proofs}\label{sec:background-zkps}
Curdleproofs is a zero-knowledge proof system, which means that it allows a prover to convince a verifier that they know a secret without revealing the secret itself.
within the context of Ethereum it could be the ability to convince someone that a transaction is valid without revealing information about the transaction such as the value of it.



\subsection{Springproofs}\label{sec:background-springproofs}
Springproofs~\cite{zhang2024springproofs} is an inner product argument that aims to allow a more flexible and efficient way of creating zero-knowledge proofs by avoiding the need for papping when working with inputs that are not of the size of power of 2.

Currently, the way to work with inner product arguments is to either only work with input sets that have the size of a power of 2, or to pad the input to the size of the next power of 2.
This leads to either forcing the prover to work with regied sizes of input sets, or to pad the input with zeros slowing down the process and forcing the prover to work with larger sets than necessary.

Springproofs is a new type of inner product argument that allows for the use of arbitrary sized input sets without the need for padding.


