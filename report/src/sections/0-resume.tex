\onecolumn     % switch to one‐column formatting
\thispagestyle{empty}

\begin{center}
    \begin{tcolorbox}[
        width=\dimexpr\paperwidth - 2in\relax,  % span from 1" margin to 1" margin
        colback=white,                          % white background
        colframe=white,                         % black border
        left=60pt, right=60pt, top=10pt, bottom=10pt, % internal padding
        boxrule=0.8pt, % thickness of border line
        fontupper=\large
    ]
        \textbf{\large Summary}%

        This paper addresses a key vulnerability in Ethereum’s Proof-of-Stake (PoS) protocol, namely the vulnerability of block proposers against Denial-of-Service (DoS) attacks.
        The attack is made possible through deanonymization, where adversaries get validator IP addresses, after which they check for matches and attack upcoming proposers.
        To counter this, Ethereum has proposed the Whisk protocol, a Single Secret Leader Election (SSLE) protocol that uses zero-knowledge proofs to hide proposer identities.
        Whisk is shuffle-based, meaning it resists adversarial tracking by iteratively shuffling a subset of a list of trackers, where each validator has a single tracker.
        Each shuffle in Whisk relies on a shuffle-based zero-knowledge proof called Curdleproofs, which uses Inner Product Arguments (IPAs) to validate the correctness of the shuffle.
        However, Curdleproofs is constrained to shuffling subsets of sizes that are only a power of two due to the recursive nature of the IPA.


        For that reason, we propose a modified protocol, CAAUrdleproofs, which lifts this restriction by integrating ideas from Springproofs.
        CAAUrdleproofs introduces a new folding scheme that allows the use of arbitrary shuffle sizes, enabling more flexibility when reducing or increasing the size.
        The experiments show that CAAUrdleproofs maintains a similar performance to Curdleproofs for power of two shuffle sizes but outperforms it when shuffle sizes are not a power of two, especially when the size is slightly above a power of two.
        This paper also proposes reducing the shuffle size from 128 down to 80, which would result in a notable decrease in the block overhead created by the protocol.
        The overhead would decrease from 16.656 KB to 12.048 KB, resulting in annual savings of approximately 12.11 GB.


        This paper also provides a security analysis of the shuffle mechanism with different shuffle sizes and under various amounts of adversarial influence.
        The results validate that smaller shuffle sizes can still maintain a secure shuffle within the total amount of shuffles available in a round of the Whisk protocol.
        The paper concludes by mentioning that CAAUrdleproofs is an efficient improvement to Curdleproofs.
        It suggests future development in the direction of post-quantum security, protocol refinement, or exploring the use of a Weighted Inner Product Argument (WIPA).



        \end{tcolorbox}
    \end{center}