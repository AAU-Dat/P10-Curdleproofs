\section{Background}\label{sec:background}
In this section, we provide the necessary background information on Ethereum and a specific attack it is vulnerable to, the Curdleproofs protocol~\cite{Curdleproofs}, the Whisk protocol~\cite{Whisk2024} and an overview of the notation used in the paper.

The notation used throughout this paper can be seen in~\autoref{tab:notation}.
\begin{table*}[!htb]
    \centering
    \begin{tabular}{|l|l|}
        \hline
        \textbf{Symbol} & \textbf{Description} \\
        \hline
        $\mathbb{G}$ & Cyclic, additive, group of prime order $p$ \\
        \hline
        $\mathbb{Z}_p$ & Ring of integers modulo $p$ \\
        \hline
        $\mathbb{G}^n,\;\mathbb{Z}_p^n$ & Vector spaces of dimension $n$ over $\mathbb{G}$ and $\mathbb{Z}_p$ \\
        \hline
        $\mathbb{Z}_p^*$ & Multiplicative group $\mathbb{Z}_p\setminus\{0\}$ \\
        \hline
        $H\in\mathbb{G}$ & Generator of $\mathbb{G}$ \\
        \hline
        $\mathbf{\gamma}\in\mathbb{Z}_p^{\lceil\log n\rceil}$ & Uniformly distributed challenges \\
        \hline
        $\mathbf{a}\in\mathbb{F}^n$ & Vector $\mathbf{a}=(a_1,\dots,a_n)\in\mathbb{F}^n$ \\
        \hline
        $\mathbf{A}\in\mathbb{F}^{n\times m}$ & Matrix with $n$ rows and $m$ columns \\
        \hline
        $\mathbf{b}=c\cdot \mathbf{a}\in\mathbb{Z}_p^n$
        & The vector where $b_i = c\,a_i$, with scalar $c\in\mathbb{Z}_p$ and $\mathbf{a}\in\mathbb{Z}_p^n$ \\
        \hline
        $\mathbf{a}\times \mathbf{b}=\sum_{i=1}^n a_i\cdot b_i$
        & Inner product of $\mathbf{a},\mathbf{b}\in\mathbb{F}^n$ \\
        \hline
        $\mathbf{G}=(g_1,\dots,g_n)\in\mathbb{G}^n,\mathbf{G'}=(g'_1,\dots,g'_n)\in\mathbb{G}^n$
        & Vectors of generators (for Pedersen commitments) \\
        \hline
        $A=a\times G=\sum_{i=1}^n a_i\cdot G_i$
        & Binding (but not hiding) commitment to $a\in\mathbb{Z}_p^n\in $ \\
        \hline
        $\mathbf{r}_A\in\mathbb{Z}^n$ & Blinding factors, e.g.\ $A=\mathbf{a}\times\mathbf{G} + \mathbf{r}_A \times \mathbf{G}$ is a Pedersen commitment to $\mathbf{a}$ \\
        \hline
        $\mathbf{a}\parallel \mathbf{b}\in\mathbb{Z}_p^{n+m}$
        & Concatenation: if $\mathbf{a}\in\mathbb{Z}_p^n$, $\mathbf{b}\in\mathbb{Z}_p^m$, then $\mathbf{a}\parallel \mathbf{b}\in\mathbb{Z}_p^{n+m}$ \\
        \hline
        $\mathbf{a}_{[:k]}=(a_1,\dots,a_k)\in\mathbb{F}^k, \; \mathbf{a}_{[k:]}=(a_{k+1},\dots,a_n)\in\mathbb{F}^{n-k}$
        & Slices of vectors (Python notation) \\
        \hline
        $\left\{\phi; w\middle|\textit{ properties satisfying }\phi,w\right\}$
        & Relation using the specified public input $phi$ and private witness $w$ \\
        \hline
    \end{tabular}
    \caption{Notation used throughout the paper.}
    \label{tab:notation}
\end{table*}


Since this work is based on the existing Curdleproofs protocol~\cite{Curdleproofs}, it inherits the same security assumptions.
Our work therefore runs as a public coin protocol in any cryptographic group where~\gls{ddh} is hard~\cite{10.1007/BFb0054851}.

\begin{definition}[DDH]
 Given a finite, multiplicative cyclic group $\mathbb{G}$ of prime order $p$, the decisional Diffie-Hellman problem is defined as follows: Given $(g^a,g^b,g^c)\in\mathbb{G}$, where $g$ is a generator of $\mathbb{G}$ and $a,b,c\in\mathbb{Z}_p$, decide whether $c=ab$.
\end{definition}

\subsection{Whisk}\label{subsec:related-work-whisk}

Ethereum uses a proof-of-stake consensus mechanism, which allows users to validate transactions and create new blocks by staking their Ether (ETH) tokens.
The Proof-of stake protocol works in epochs of 32 slots, where each slot is 12 seconds long.
In each slot a proposer is chosen to propose a block thereby allowing the network to reach consensus on the state of the blockchain.

The proposer DoS attack is a type of attack that targets the block proposers making them unable to propose blocks.
An adversary can use the proposer DoS attack to prevent a proposer from receiving rewards, gotten from proposing a block, and increase their own rewards~\cite{EthereumSSLE2024}.
As a response to the proposer DoS attack, Ethereum has proposed a new protocol called Whisk~\cite{Whisk2024} as an attempt to mitigate the attack.
An attack on the Ethereum network that was discovered by Heimbach et al.~\cite{heimbach2024deanonymizingethereumvalidatorsp2p} is the deanonymization attack on validators.
In our preliminary work~\cite{ouroldpaper}, we have shown that the attack is still possible to perform on the Ethereum network, and using the attack, a proposer~\gls{dos} can be performed.


Whisk is a zero-knowledge Single Secret Leader Election (SSLE) system that uses a zero-knowledge argument called Curdleproofs~\cite{Curdleproofs} to verify the correctness of a shuffle without revealing the input or output~\cite{10.1145/3419614.3423258}.
Whisk works by selecting a list of 16,384 validator trackers and shuffles them over 8,192 slots ($\sim$1 day).
Then 8,192 proposers are selected from the shuffled list to propose blocks for the next 8,192 slots while a new list is being shuffled.
This way a new list of proposers is created every day.
After each shuffle, Whisk uses a zero-knowledge proof to prove that the shuffle is correct.
This is so that the proposer can prove that they are the correct proposer for the slot without revealing their identity, thereby mitigating the proposer~\gls{dos} attack because of the identity of the upcoming proposers being hidden now.

Curdleproofs is a zero-knowledge proof system that allows a prover to prove knowledge of a shuffle without revealing how it shuffled the elements.
It does so by using three different zero-knowledge proofs, with one of them relying on two more zero-knowledge proofs.

The first proof is the~\gls{sameperm} proof.
The prover first constructs a commitment to the permutation, $\sigma()$, by saying $M=\sigma(1,2,\dots,\ell)\times\mathbf{g}$.
Then, using the Fiat-Shamir transformation, a challenge, $\mathbf{a}$, from public inputs is constructed, and a new commitment is made from that, $A=\sigma(\mathbf{a})\times\mathbf{g}$.
The~\gls{sameperm} proof now consists of convincing the verifier that the same permutation was used for constructing commitment $A$ and $M$.
To do this, the two commitments are used to construct a polynomial equation.
Then Neff's trick~\cite{10.1145/501983.502000} is used, which observes that two polynomials are equal iff.\ their roots are the same up to permutation.

To prove that, the protocol makes use of a grand product argument.
To prove that argument, Curdleproofs compiles it down to an~\gls{ipa} by expressing each multiplication of the grand product as its own equation.

Hence, the~\gls{sameperm} proof is done if the prover can prove the~\gls{ipa}.


The second proof is a~\gls{samemsm} argument.
The prover should by now have proven the existence of the permutation.
Now, the goal of the~\gls{samemsm} argument is to prove that the output ciphertext set was constructed with the same permutation, here called multiscalar, committed to in commitment $A$.
As the multiscalar is a vector this argument is an~\gls{ipa} by nature, contrary to the~\gls{sameperm} argument.

The third proof is a Same Scalar argument.
To mask the ciphertexts, each prover, besides permuting the set, multiplies all ciphertexts by a scalar, $k$.
This is for randomization purposes, making it harder for adversaries to track the ciphertexts~\cite{Whisk2024}.
Also, all validators are still able to open their commitments if they are chosen as block proposers, even after several randomizations.
So, the goal of the Same Scalar argument is to prove the existence of the scalar, $k$, such that the commitment of the permuted set is equal to the commitment of the pre-permuted set multiplied by $k$.



\subsection{Zero-knowledge proofs}\label{sec:background-zkps}
Curdleproofs is a zero-knowledge proof system, which means that it allows a prover to convince a verifier that they know a secret without revealing the secret itself.
Within the context of Ethereum it could be the ability to convince someone that a transaction is valid without revealing information about the transaction such as the value of it.
In Whisk, it uses Curdleproofs to prove the validity of a shuffle.

\begin{definition}[Zero-Knowledge Argument of Knowledge]
    An argument $(Setup, P, V)$ is a zero-knowledge argument of knowledge of a relation $\mathbb{R}$ if it satisfies completeness, knowledge-soundness and is honest-verifier zero-knowledge.
\end{definition}

Definitions for knowledge-soundness, completeness, and~\gls{hvzk} can be found in~\autoref{sec:appendix}.

\subsection{Problem definition}\label{subsec:problem-definition}
In Chapter 6 of Curdleproofs~\cite{Curdleproofs}, they explain the efficiency of the protocol, including also the size of the proof.
They specifically mention that the proof has size $18+10 \log(\ell+4)\mathbb{G}$, $7\mathbb{F}$.
As the proof size is dependent on the size of the shuffle, $\ell$, an interest in the possibility of reducing this parameter arises.
The current proposal of curdleproofs only works on shuffles, where the size is a power of 2.
The reason is that the underlying proofs, such as the~\gls{ipa}, needs to fold recursively down to 1, by halving the size in every round.

