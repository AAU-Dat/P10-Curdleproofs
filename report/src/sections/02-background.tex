
\section{Background}\label{sec:background}
In this section, we will go through some of the concepts that will be used in the rest of the paper as well as some surrounding context like attacks performed.

\subsection{Ethereum and Proof of Stake}\label{subsec:ethereum-and-proof-of-stake}
Ethereum is a blockchain platform that allows developers to create decentralized applications using smart contracts.
Previously operating with a~\gls{pow} consensus algorithm, Ethereum transitioned to a~\gls{pos} consensus algorithm in 2022.
This transition was done to reduce the energy consumption of the network and to increase the scalability of the network.
The transition was done in a series of upgrades called the Ethereum 2.0 upgrade.
~\gls{pos} is a consensus algorithm that is used to secure blockchain net by helping to create new blocks and confirm transactions.
It works by creating validators based on the amount of cryptocurrency they have staked then it selects som of these validators as proposers to be the ones to create new blocks which then are confirmed by the rest of the validators.
The proposers then get rewarded for creating a valid block and the validators get rewarded for confirming the block.
Block proposing happens within epochs of 32 blocks per epochs and for each epoch a group of validators is selected and from them a proposer is chosen.
the way the proposer is chosen is through a random proses that is weighted by the amount of cryptocurrency the validator has staked and using the publicly available \gls{randao} algorithm to simulate the random selection.
The blocks also have a time limit of 12 seconds to be created and confirmed else the block is discarded and the proposer is penalized.
If a fork happens the validators have to choose which fork to follow this is done by using the \gls{lmd-ghost} algorithm which chooses the fork with the greatest weight of attestations in its history~\cite{EthereumProof-of-stakePoS}.


\subsection{Zero-Knowledge Proofs}\label{subsec:zero-knowledge-proofs}
A~\gls{zkp} is a cryptographic method that allows one party to prove to another party that something is true without revealing any information.

Two of the subcategories of \gls{zkp}s are \gls{zk-snark} and \gls{zk-stark}.
~\gls{zk-snark} is the more common of the two.
It uses elliptic curve cryptography to create proofs based on the assumption that it is hard to find the discrete logarithm form the publicly known base point.
~\gls{zk-snark} has a start-up ritual which requires a trusted setup by all parties involved that the original proving key is destroyed as to not be able to create fake proofs.
A criticism of \gls{zk-snark} is that it is not quantum resistant because of the reliance on elliptic curve cryptography.

~\gls{zk-stark} on the other hand, is a newer and more complex~\gls{zkp}.
Despite not having non-interactive in its name,~\gls{zk-stark} is also non-interactive.
Different form~\gls{zk-snark},~\gls{zk-stark} uses hashing functions to create proofs instead of elliptic curves.
This method is post-quantum secure and does not require a trusted setup.
It does, however, come with a higher computational cost and is not as widely used and documented as~\gls{zk-snark}.
