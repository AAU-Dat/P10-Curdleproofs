\section{Background}\label{sec:background}
In this section, we provide the necessary background information on Ethereum and a specific attack it is vulnerable to, the Whisk protocol~\cite{Whisk2024}, and the Curdleproofs protocol~\cite{Curdleproofs} used in Whisk.

The notation used throughout this paper can be seen in~\autoref{tab:notation}.
\begin{table*}[!htb]
    \centering
    \begin{tabular}{|l|l|}
        \hline
        \textbf{Symbol} & \textbf{Description} \\
        \hline
        $\mathbb{G}$ & Cyclic, additive, group of prime order $p$ \\
        \hline
        $\mathbb{Z}_p$ & Ring of integers modulo $p$ \\
        \hline
        $\mathbb{G}^n,\;\mathbb{Z}_p^n$ & Vector spaces of dimension $n$ over $\mathbb{G}$ and $\mathbb{Z}_p$ \\
        \hline
        $\mathbb{Z}_p^*$ & Multiplicative group $\mathbb{Z}_p\setminus\{0\}$ \\
        \hline
        $H\in\mathbb{G}$ & Generator of $\mathbb{G}$ \\
        \hline
        $\mathbf{\gamma}\in\mathbb{Z}_p^{\lceil\log n\rceil}$ & Uniformly distributed challenges \\
        \hline
        $\mathbf{a}\in\mathbb{F}^n$ & Vector $\mathbf{a}=(a_1,\dots,a_n)\in\mathbb{F}^n$ \\
        \hline
        $\mathbf{A}\in\mathbb{F}^{n\times m}$ & Matrix with $n$ rows and $m$ columns \\
        \hline
        $\mathbf{b}=c\cdot \mathbf{a}\in\mathbb{Z}_p^n$
        & The vector where $b_i = c\,a_i$, with scalar $c\in\mathbb{Z}_p$ and $\mathbf{a}\in\mathbb{Z}_p^n$ \\
        \hline
        $\mathbf{a}\times \mathbf{b}=\sum_{i=1}^n a_i\cdot b_i$
        & Inner product of $\mathbf{a},\mathbf{b}\in\mathbb{F}^n$ \\
        \hline
        $\mathbf{g}=(g_1,\dots,g_n)\in\mathbb{G}^n,\mathbf{g'}=(g'_1,\dots,g'_n)\in\mathbb{G}^n$
        & Vectors of generators (for Pedersen commitments) \\
        \hline
        $A=a\times G=\sum_{i=1}^n a_i\cdot G_i$
        & Binding (but not hiding) commitment to $a\in\mathbb{Z}_p^n\in $ \\
        \hline
        $\mathbf{r}_A\in\mathbb{Z}^n$ & Blinding factors, e.g.\ $A=\mathbf{a}\times\mathbf{g} + \mathbf{r}_A \times \mathbf{g}$ is a Pedersen commitment to $\mathbf{a}$ \\
        \hline
        $\mathbf{a}\parallel \mathbf{b}\in\mathbb{Z}_p^{n+m}$
        & Concatenation: if $\mathbf{a}\in\mathbb{Z}_p^n$, $\mathbf{b}\in\mathbb{Z}_p^m$, then $\mathbf{a}\parallel \mathbf{b}\in\mathbb{Z}_p^{n+m}$ \\
        \hline
        $\mathbf{a}_{[:k]}=(a_1,\dots,a_k)\in\mathbb{F}^k, \; \mathbf{a}_{[k:]}=(a_{k+1},\dots,a_n)\in\mathbb{F}^{n-k}$
        & Slices of vectors (Python notation) \\
        \hline
        $\left\{\phi; w\middle|\textit{ properties satisfying }\phi,w\right\}$
        & Relation using the specified public input $phi$ and private witness $w$ \\
        \hline
    \end{tabular}
    \caption{Notation used throughout the paper.}
    \label{tab:notation}
\end{table*}


Since this work is based on the existing Curdleproofs protocol~\cite{Curdleproofs}, it inherits the same security assumptions.
Our work therefore runs as a public coin protocol in any cryptographic group where~\gls{ddh} is hard~\cite{10.1007/BFb0054851}.
\gls{ddh} is defined as follows.

\begin{definition}[DDH]
 Given a finite, multiplicative cyclic group $\mathbb{G}$ of prime order $p$, the decisional Diffie-Hellman problem is defined as follows: Given $(g^a,g^b,g^c)\in\mathbb{G}$, where $g$ is a generator of $\mathbb{G}$ and $a,b,c\in\mathbb{Z}_p$, decide whether $c=ab$.
\end{definition}

\subsection{Zero-knowledge proofs}\label{sec:background-zkps}
Before explaining the protocol, we must mention that Curdleproofs, and hence also Whisk, is a~\gls{zkp} system.
It is a system that allows a prover to convince a verifier that they know a secret without revealing the secret itself.
Within the context of Ethereum, it could be the ability to convince someone that a transaction is valid without revealing information about the transaction such as the value of it.
Whisk uses Curdleproofs to prove the validity of a shuffle.

\begin{definition}[Zero-Knowledge Argument of Knowledge]
    An argument $(Setup, P, V)$ is a zero-knowledge argument of knowledge of a relation $\mathbb{R}$ if it satisfies completeness, knowledge-soundness and is honest-verifier zero-knowledge.
\end{definition}

Definitions for knowledge-soundness, completeness, and~\gls{hvzk} can be found in Appendix~\ref{sec:appendix}.

Also, two of three proofs that make up Curdleproofs are~\glspl{ipa}.
These are also~\glspl{zkp}, and will be the focus of this paper.
Hence, we provide a definition on~\glspl{ipa}.

\begin{definition}[Inner Product Argument]
    The argument takes as input two binding vector commitments $C=\mathbf{c}\times\mathbf{g}\in\mathbb{G}$ and $D=\mathbf{d}\times\mathbf{g'}\in\mathbb{G}$ to the vectors $\mathbf{c},\mathbf{d}\in\mathbb{Z}_p^n$ and $z\in\mathbb{Z}_p$.
    The goal is to prove that $z=\mathbf{c}\times\mathbf{d}$.
    The argument has logarithmic communication by halving the dimensions of $\mathbf{c}$ and $\mathbf{d}$ in each iteration.
\end{definition}

\subsection{Whisk}\label{subsec:related-work-whisk}
Ethereum uses a~\gls{pos} consensus mechanism, which allows users to validate transactions and create new blocks by staking their~\gls{eth} tokens.
The~\gls{pos} protocol works in epochs of 32 slots, where slots are 12 seconds long.
In each slot a proposer is chosen to propose a block thereby allowing the network to reach consensus on the state of the blockchain.

The proposer~\gls{dos} attack is a type of attack that targets the block proposers, making them unable to propose blocks.
An adversary can use the proposer~\gls{dos} attack to prevent a proposer from receiving rewards, gotten from proposing a block, and increase their own rewards~\cite{EthereumSSLE2024}.
As a response to the proposer~\gls{dos} attack, Ethereum proposed a new protocol called Whisk~\cite{Whisk2024} as an attempt to mitigate the attack.
An attack on the Ethereum network that was discovered by Heimbach et al.~\cite{heimbach2024deanonymizingethereumvalidatorsp2p} is the deanonymization attack on validators.
In our preliminary work~\cite{ouroldpaper}, we show that the attack is still possible to perform on the Ethereum network, and using the attack, a proposer~\gls{dos} can be performed.


Whisk is a~\gls{zk}~\gls{ssle} system that uses a~\gls{zk} argument called Curdleproofs~\cite{Curdleproofs} to verify the correctness of a shuffle with size $\ell$ without revealing the input or output~\cite{10.1145/3419614.3423258}.
Whisk works by selecting a list of 16,384 validator trackers and shuffles them over 8,192 slots ($\sim$1 day).
Then 8,192 proposers are selected from the shuffled list to propose blocks for the next 8,192 slots while a new list is being shuffled.
This way a new list of proposers is created every day.
After each shuffle, Whisk uses a~\gls{zkp} to prove that the shuffle is correct.
This is so that the proposer can prove that they are the correct proposer for the slot without revealing their identity, thereby mitigating the proposer~\gls{dos} attack because of the identity of the upcoming proposers being hidden now.

Curdleproofs is a~\gls{zkp} system that allows a prover to prove knowledge of a shuffle without revealing how it shuffled the elements.
It does so by using three different~\glspl{zkp}, with one of them relying on two more~\glspl{zkp}.
The overview can be seen in~\autoref{fig:curdleproof-protocol}.

\begin{figure}[!ht]
    \centering
    \begin{circuitikz}[scale = 0.8, transform shape]
        \tikzstyle{every node}=[font=\normalsize]
        \draw[rounded corners]  (4,11.75) rectangle (7.75,10.75);
        \node  at (5.75,11.25) {$\mathbf{R}$, $\mathbf{S}$, $\mathbf{T}$, $\mathbf{U}$, $M$};
        \node  at (3.2,11.25) {Input};
        \draw [->, >=Stealth] (5.75,10.75) -- (5.75,10.25);
        \draw[rounded corners]  (4,10.25) rectangle (7.75,9.25);
        \node  at (5.875,9.75) {$T=\sigma(k\mathbf{R})$, $U=\sigma(k\mathbf{S})$};
        \node  at (2.875,9.75) {Statement};
        \draw [->, >=Stealth] (5.75,9.25) -- (5.75,8.75);
        \draw[rounded corners]  (4,8.75) rectangle (7.75,7.75);
        \node  at (5.75,8.25) {$\mathbf{a\leftarrow}$Fiat-Shamir};
        \draw [->, >=Stealth] (5.75,7.75) -- (5.75,7.25);
        \draw[rounded corners]  (3.75,7.25) rectangle (8,5.75);
        \node  at (5.75,6.75) {A=$\sigma(\mathbf{a})\times \mathbf{g}$};
        \node  at (5.875,6.25) {$T=\mathbf{a}\times k\mathbf{R}$, $U=\mathbf{a}\times k\mathbf{S}$};
        \draw [->, >=Stealth] (5.75,5.75) -- (2.75,5);
        \draw [->, >=Stealth] (5.75,5.75) -- (5.75,5);
        \draw [->, >=Stealth] (5.75,5.75) -- (8.75,5);
        \draw[fill=red, fill opacity=0.3, rounded corners]  (0.6,5) rectangle (4.25,2.75);
        \draw[fill=green, fill opacity=0.3, rounded corners]  (7.25,5) rectangle (9.75,2.75);
        \draw[fill=blue, fill opacity=0.3, rounded corners]  (4.5,5) rectangle (7,2.75);
        \node [font=\large] at (2.4,4.5) {SamePerm};
        \node  at (2.4,4) {A=$\sigma(\mathbf{a})\times \mathbf{g}$};
        \node  at (2.4,3.5) {$M=\sigma(1,2,\dots,\ell)\times \mathbf{g}$};
        \node [font=\large] at (5.75,4.5) {SameMSM};
        \node  at (5.75,4) {$A=\mathbf{v}\times \mathbf{g}$};
        \node  at (5.75,3.5) {$T=\mathbf{v}\times \mathbf{T}$};
        \node  at (5.75,3) {$U=\mathbf{v}\times \mathbf{U}$};
        \node [font=\large] at (8.5,4.5) {SameScalar};
        \node  at (8.5,4) {$T=k(\mathbf{a}\times \mathbf{R})$};
        \node  at (8.5,3.5) {$U=k(\mathbf{a}\times \mathbf{S})$};
        \draw [->, >=Stealth] (2.4,2.75) -- (2.4,2.5);
        \draw[fill=red, fill opacity=0.3, rounded corners]  (0.6,2.5) rectangle (4.25,1);
        \node [font=\large] at (2.4,2) {GrandProd};
        \node  at (2.4,1.5) {$p=\Pi_{i=1}^\ell b_i$};
        \draw [->, >=Stealth] (4.25,1.75) -- (4.5,1.75);
        \draw[fill=red, fill opacity=0.3, rounded corners]  (4.5,2.5) rectangle (7,1);
        \node [font=\large] at (5.75,2) {DL IPA};
        \node  at (5.75,1.5) {$z=\mathbf{c}\times\mathbf{d}$};
    \end{circuitikz}

    \caption{Overall structure of the Curdleproofs protocol. Modified figure from~\cite{Curdleproofs}.}
    \label{fig:curdleproof-protocol}
\end{figure}

The first proof is the~\gls{sameperm} proof.
The prover first constructs a commitment to the permutation,~$\sigma()$, by saying $M=\sigma(1,2,\dots,\ell)\times\mathbf{g}$, where~$\ell$ is the number of shuffled trackers, and $\mathbf{g}$ is a vector of cryptographic generators.
Then, using the Fiat-Shamir transformation, a challenge,~$\mathbf{a}$, from public inputs is constructed, and a new commitment is made from that, $A=\sigma(\mathbf{a})\times\mathbf{g}$.
The~\gls{sameperm} proof consists of convincing the verifier that the same permutation was used for constructing the commitments $A$ and $M$.
To do this, the two commitments are used to construct a polynomial equation.
Then Neff's trick~\cite{10.1145/501983.502000} is used, which observes that two polynomials are equal iff.\ their roots are the same up to permutation.

In order to show this, the protocol makes use of a~\gls{grandprod} argument.
To prove that argument, Curdleproofs compiles it down to a~\gls{dlipa} by expressing each multiplication of the grand product as its own equation.
The proof of the~\gls{dlipa} then stems from the protocol originally proposed by Bootle et al.~\cite{cryptoeprint:2016/263,Curdleproofs}

Hence, the~\gls{sameperm} proof is done if the prover can prove the~\gls{dlipa}.


The second proof is a~\gls{samemsm} argument.
The prover has proven the existence of the permutation.
Now, the goal of the~\gls{samemsm} argument is to prove that the output ciphertext set was constructed with the same permutation, $\sigma$, here called multiscalar $\mathbf{v}$\footnote{Denoted as $\mathbf{c}$ in the Curdleproofs paper but changed for readability}, committed to in commitment $A$.
Note, therefore, that $A$ in~\gls{sameperm} and~\gls{samemsm} is the same commitment, where $\mathbf{v}=\sigma(\mathbf{a})$
As the multiscalar is a vector, this argument is an~\gls{ipa} by nature, contrary to the~\gls{sameperm} argument.

The third proof is a Same Scalar argument.
To mask the ciphertexts, each prover, besides permuting the set, multiplies all ciphertexts by a scalar, $k$.
This is for randomization purposes, making it harder for adversaries to track the ciphertexts~\cite{Whisk2024}.
Also, all validators are still able to open their commitments if they are chosen as block proposers, even after several randomizations.
Therefore, the goal of the Same Scalar argument is to prove the existence of the scalar,~$k$, such that the commitment of the permuted set is equal to the commitment of the pre-permuted set multiplied by $k$.


In Chapter 6 of Curdleproofs~\cite{Curdleproofs} they explain that the proof has size~$18+10 \log(\ell+4)\mathbb{G}$, $7\mathbb{F}$, where $\mathbb{G}$ is a cryptographic group point, and $\mathbb{F}$ is a field element.

\subsection{Problem definition}\label{subsec:problem-definition}
The current proposal of Curdleproofs only works when the shuffle size of Whisk is set to a power of 2.
The reason is that the underlying proofs,~\gls{dlipa} in~\gls{sameperm} and~\gls{samemsm}, need to fold recursively down to 1, by halving the size in every round.
With the current shuffling size of 128, being able to choose the size more flexibly could lead to both performance and size gains.
The problem we study in this article is therefore how to extend Curdleproofs to~$\ell$ values that are not a power of 2.

