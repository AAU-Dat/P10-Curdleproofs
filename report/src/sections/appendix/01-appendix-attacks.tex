
\section{Attacks on Ethereum}\label{sec:attacks-on-ethereum}
In the following section is an explanation of different attacks
that either can or could happen to the ethereum blockchain.
There is a description of both the attack and its possible mitigation.
If found, there will be a link to a~\gls{poc} of the attack and/or the proposed/implemented mitigation.
\subsection{Reorg}\label{subsec:reorg}
One of the reorganization attacks is \textit{Commitment Attacks on LMD GHOST}~\cite{sarenche2024breakingbalancepowercommitment}.
It works by using financial incentives to convince validators to vote for a prior block in the chain, by saying it is the block you will be commiting to, to try and exploit there laziness.
Possible because around 90\% of validators use software called "MEV Boost", which tries to earn you the most money.

Another reorg attack is Short-range reorg which uses short-range reorgs of the blockchain stipulating consensus to delay finality of consensus decisions.
Such short-range reorgs also allow validators to increase their earnings from participating in the protocol.
It does this by withholding a block and then releases it timed with the next honest block in order to orphan it.
This attack requires large amount of stake to be hold by the adversary with 30\% being the aim and everything below only reducing the chance of succes~\cite{10.1007/978-3-031-18283-9_28}.

A reorg attack that has been inspired bt the last attack is Low cost long-range reorg attack.
This attack works by the adversary avoids competing directly with honest validators of $(k - 1)$ committees, as done in the short-range reorg attack.
Instead, the adversary uses the technique of balancing attacks to keep honest committee members split roughly in half by ensuring they have different views on what the current head of the chain is.
This makes honest nodes work against each other to maintain a tie which the adversary can tip to their liking at any point using only a few votes.
A possible mitigation is the recently proposed Transaction Encapsulation approach demonstrates that some cross-layer attacks can be alleviated or prevented by dynamic transaction~\cite{10.1007/978-3-031-18283-9_28}.

A different reorg attack is called $>=50\%$ stake attack.
The attack works if the adversary has over 50\% stake, the adversary would be able to have the majority vote every time in the fork choice algorithm.
This would mean that all honest validators need to vote along with the adversary to not get their ETH burned due to the inactivity leak security protocol.
Because of said protocol, the adversary would also eventually gain finalization.
This attack also makes it possible for the adversary to easily perform a reorg of a block~\cite{EthereumAttackDefense2024}.
It is thought that given the large amount of stake needed for the attack to work, this attack would not be feasible.



\subsection{DoS}\label{subsec:dos}
We've found three different kinds of DoS attacks that either were or are possible to perform on Ethereum.

One of the attacks is called \textit{under-priced opcodes}~\cite{10.1145/3391195,9815256}.
This attack works because Ethereum has a gas mechanism to reduce abuse of computing resources.
Though when a contract has a lot of underpriced opcodes, they will consume many resources.
Execution of contracts requires a lot of resources.

To mitigate this,
Ethereum has raised the gas cost of opcodes
to preserve the number of transactions-per-second~\cite{Opcode-mitigation}~\footnote{
\href{https://github.com/ethereum/EIPs/blob/master/EIPS/eip-150.md}{EIPs/EIPS/eip-150.md at master ethereum/EIPs GitHub}}.

Another attack,
which is closely related to the former, is \textit{empty account in the state trie}~\cite{10.1145/3391195,9815256}.
This attack was possible
because the existence of empty accounts increases the transaction processing time and synchronization.
An empty account is an account with zero balance and no code.
The attack required the proposer to select only the transactions of the adversary,
which could be insured by offering a higher gas price.

The mitigation is a combination of the one
explained for \textit{under-priced opcodes} as well as a mitigation
for clearing empty accounts~\cite{Opcode-mitigation,empty-account-mitigation,empty-account-eip-mitigation}~\footnote{
\href{https://github.com/ethereum/EIPs/blob/master/EIPS/eip-161.md}{EIPs/EIPS/eip-161.md at master ethereum/EIPs GitHub}}.

The last example of a DoS attack is called \textit{Proposer DoS}~\cite{EthereumSSLE2024,EthereumAttackDefense2024}.
The background to making this attack possible is
that the consensus mechanism uses a publicly known function for choosing the upcoming block proposers.
The adversary is therefore able to compute this in slight advance of the blockchain, s.t.\ each proposer is now known.
After this, the adversary can map the proposer's IP addresses and overload their connection.
A successful attack would leave a proposer unable to propose their block in time.

To prevent this kind of attack,
Ethereum plans
to use something they call~\gls{ssle} which ensures
that only the selected validator knows that they have been selected~\cite{EthereumSSLE2024,EthereumResearchSSLE2024}.

Specifically, a proposal has been made to use an election protocol called Whisk, which is a type of~\gls{ssle}~\cite{Whisk2024}\footnote{\href{https://github.com/ethresearch/Shuffle_SSLE/tree/master/rust_code/src}{Shuffle\_SSLE/rust\_code/src at master ethresearch/Shuffle\_SSLE GitHub}}\footnote{\href{https://github.com/ethereum/consensus-specs/pull/2800}{[WIP] Introduce consensus code for Whisk (SSLE) by asn-d6 Pull Request \#2800 ethereum/consensus-specs GitHub}}\footnote{\href{https://github.com/dapplion/lighthouse/tree/whisk}{GitHub - dapplion/lighthouse at whisk}}.
It works by each validator submitting a commitment to a secret shared by all validators.
The commitments are shuffled s.t.\ no-one can map commitments to the validators,
but each validator knows what commitment belongs to them.
This shuffle-phase goes on for a day, 256 epochs, before using the shuffled proposer list the following day.
Commitments are chosen at random, and the selected proposer will detect its commitment to know when to propose a block.

The shuffling phase requires validators to occasionally shuffle a subset of candidate proposers.
Using a subset is a measure to reduce computation for the validators, as 256 epochs correspond to 8912 proposers.
The shuffle requires a validator to construct a~\gls{zkp} to confirm that the shuffle was performed correctly.

\subsection{Balancing Attack}\label{subsec:balancing-attack}
For this type, we found two attacks.

The first attack we call \textit{LMD-specific balancing attack}~\cite{10.1145/3560829.3563560}.
This attack exploits the~\gls{lmd} \textit{proposer
boosting} by sending out two competing blocks
but giving half the validators one block before the other and the opposite for the other half.
This would create a fork in the blockchain.

Although no mitigation is mentioned,
it requires $W_p/b+1$ adversarial slots~\footnote{where $W_p$ is the proposer boost weight (fx 100 validators/slot: $W_p=0.7\cdot100=70$)
    and $b$ is the fraction of adversaries in the committee in each slot}.


Another balancing attack has the adversary
exploiting adversarial network delay
and strategic voting by a vanishing fraction of adversarial validators
to stall the protocol indefinitely~\cite{10.1007/978-3-031-18283-9_28}~\footnote{
\href{https://github.com/tse-group/gasper-gossip-attack}{GitHub - tse-group/gasper-gossip-attack}}.

Though this attack does depend on networking assumptions that are highly contrived in practice;
those being the attacker having fine-grained control over latencies of individual validators.

\subsection{Finality Attack (Bouncing Attack)}\label{subsec:finality-attack-(bouncing-attack)}
For the Ethereum, there exist attacks called finality attacks, also known as bouncing attacks,
which have the purpose of denying the blockchain to finalize its block, halting its functionality.

The first attack is a \textit{double finality} attack~\cite{EthereumAttackDefense2024, 10646904}.
It is theoretically possible for an attacker that wants to risk 34\% of the total staked ether.
Two forks finalize simultaneously, creating a permanent split of the chain.

A way to see a mitigation of this is
that it is practically impossible given the current value of 34\% of the total staked ether.
Also, voting on two different chains, called double voting, is a slash-able offense in the Ethereum chain,
so the adversary would get their staked ETH slashed.


Another finality attack is called \textit{33\% finality attack}~\cite{EthereumAttackDefense2024}.
Here, all adversarial ($\geq33\%$) validators can simply go inactive,
meaning that a block cannot get $2/3$ attestations which is required to achieve finality.
Therefore, the blockchain would not be able to go further, as it would not be able to achieve finality.

The mitigation for this is called \textit{the inactivity leak}.
The Ethereum chain penalizes the validators who resist voting or are inactive.
Their ETH gets burned until the majority vote has a 2/3 majority.
This makes the attack impractical for the attacker,
as it is costly to have $\geq33\%$ of the total staked ETH and their ETH gets burned as well.

\subsection{Avalanche Attack}\label{subsec:avalanche-attack}
This type only includes a single attack that we call \textit{avalanche attack on proof-of-stake ghost}~\cite{10.1145/3560829.3563560}~\footnote{
\href{https://github.com/tse-group/pos-ghost-attack?tab=readme-ov-file}{GitHub - tse-group/pos-ghost-attack}}.
The attack uses withheld blocks to make wide subtrees to displace an honest chain.
It does this by exploiting the reuse of uncle blocks.

The mitigation for this attack is already a part of the Ethereum blockchain~\footnote{\href{https://github.com/ethereum/consensus-specs/pull/2730}{Proposer LMD Score
Boosting by adiasg Pull Request \#2730 ethereum/consensus-specs GitHub}}.
It is mitigated by~\gls{lmd}, which works together with~\gls{ghost}.
The protocol only counts a vote from a validator if the vote is strictly later than the current entry.
If two equivocating votes were sent from the same validator at the same time slot,
only the earlier message would be counted.

\subsection{Bribery}\label{subsec:bribery}
An adversary using bribery attacks on the Ethereum chain could be interested
in dictating a choice in some sort of voting mechanism.
This is exactly what happens in a described \textit{quadratic funding} attack~\cite{EthereumAttacks2024}.
The adversary researches votes on the chain and bribes users to vote for what the adversary wants.

In defense of a potential bribery attack,
the Ethereum blockchain implements a private voting system called~\gls{maci}~\cite{EthereumAttacks2024,MACI2022}~\footnote{
\href{https://github.com/privacy-scaling-explorations/maci}{GitHub -
privacy-scaling-explorations/maci: Minimal Anti-Collusion Infrastructure (MACI)}}.

What ~\gls{maci} does is essentially hiding what each person has voted for.
It does so by demanding the voters to send their votes encrypted to a central coordinator.
This coordinator constructs~\gls{zk-snark} proofs, which verifies that all messages were processed correctly,
and that the final result corresponds to the sum of all valid votes.

As votes are now hidden, the adversary is not able, by oneself, to prove that the bribee voted in way of said bribery.
Though the bribee could decrypt their own message and show the vote to the adversary.

~\gls{maci} has fixed this problem by implementing public key switching.
This means that a voter can request a new public key.
In addition to this, a vote is only valid if it uses the most recent public key of the voter.
Therefore, a bribee can show its first vote obeying the adversary,
generate a new public key, and send a new, now honest, vote.
The old vote will then become invalid as it uses a deprecated public key.

\subsection{Staircase Attack}\label{subsec:staircase-attack}
The last type of attack that we cover is called a \textit{staircase attack}.
This is a two-part attack, with a warm-up attack and a full attack~\cite{cryptoeprint:2023/1622}~\footnote{
\href{https://github.com/tsinghua-cel/Staircase-Attack}{GitHub - tsinghua-cel/Staircase-Attack: This is the staircase attack implement for the paper "Max Attestation Matters: Making Honest Parties Lose Their Incentives in Ethereum PoS."}}.

First, we will cover the \textit{warm-up attack}.
The attack works when the adversary, called $a$, has to propose the first block of an epoch.
From there, the attack works in four parts:
\begin{enumerate}
    \item The adversary withholds its block at slot $t$
    \item The honest attestors in slot $t$ create attestation with the last block of the previous epoch, called $b$.
    \item Adversary $a$ releases block $b_t$ and honest validators update their checkpoint to block $b_t$.
    \item Attestations with targets different from $b_t$ will be discarded,
    and honest attestors in slot $t$ will be penalized eventually
\end{enumerate}

This attack is mitigated by what Ethereum calls \textit{honest reorg},
which should prevent intentionally withheld blocks~\cite{cryptoeprint:2023/1622}.

A block proposed in slot t with fewer than 20\% attestations is considered invalid.
Though it can be theoretically avoided with network timing, ensuring at least 20\% of the attestors get the block.


With the \textit{warm-up attack}, the author goes on to describe a full \textit{staircase attack}.
The attack starts with the \textit{warm-up attack}.
Then the plan is to manipulate the source of the honest validators.
Half the validators should get an outdated \textit{last justified checkpoint} and be penalized.
All byzantine validators withhold their blocks and publish them in the middle of the epoch.
It should be able
to be done as a one-time attack with probability 98.84\%
if adversary controls N/3 validators~\cite{cryptoeprint:2023/1622}.
The byzantine validators don't get penalized, but will get a smaller reward than the fair share.

What would make this attack infeasible is that it would require a lot of controlled validators.
The author states that for Ethereum to be vulnerable, there would need to be $<16384$ validators,
which makes the attack as good as impossible given Ethereum has $~1.073.406$ daily active validators~\cite{cryptoeprint:2023/1622}~\footnote{According to \href{https://beaconcha.in/charts/validators}{beaconcha.in} as of 22-10-2024}.
Also,
Ethereum released a patch
fixing this attack after the author pointed it out~\footnote{\href{https://github.com/ethereum/consensus-specs/pull/3339\#issuecomment-1637117341}{Confirmation Rule by saltiniroberto Pull Request \#3339 ethereum/consensus-specs GitHub}}.