
\section{Attacks on Ethereum}\label{sec:attacks-on-ethereum}
\subsection{Reorg}\label{subsec:reorg}
One of the reorganization attacks is \textit{Commitment Attacks on LMD GHOST}~\cite{sarenche2024breakingbalancepowercommitment}.
It works by using financial incentives to convince validators to vote for a prior block in the chain, by saying it is the block you will be commiting to, to try and exploit there laziness.
Possible because around 90\% of validators use software called "MEV Boost", which tries to earn you the most money.

Another reorg attack is Short-range reorg which uses short-range reorgs of the blockchain stipulating consensus to delay finality of consensus decisions.
Such short-range reorgs also allow validators to increase their earnings from participating in the protocol.
It does this by withholding a block and then releases it timed with the next honest block in order to orphan it.
This attack requires large amount of stake to be hold by the adversary with 30\% being the aim and everything below only reducing the chance of succes.





\subsection{DoS}\label{subsec:dos}

\subsection{Balancing Attack}\label{subsec:balancing-attack}

\subsection{Finality Attack (Bouncing Attack)}\label{subsec:finality-attack-(bouncing-attack)}

\subsection{Avalanche Attack}\label{subsec:avalanche-attack}

\subsection{Bribery}\label{subsec:bribery}

\subsection{Staircase Attack}\label{subsec:staircase-attack}