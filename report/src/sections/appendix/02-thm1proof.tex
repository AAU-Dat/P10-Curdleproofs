
\section{Proof of Theorem 1}\label{sec:appendix-thm1proof}
\begin{proof}
    The following proof is divided into two separate proofs.
    First we prove~\gls{hvzk}.

    After this, both knowledge-soundness and completeness are proven in the same proof.

    \paragraph*{\textbf{Proof of HVZK}}
    CAAUrdleproofs is the Curdleproofs DL~\gls{ipa} on which the Springproofs protocol has been applied.
    So to help show that it is~\gls{hvzk}, we refer to Theorem 5 of Springproofs~\cite{zhang2024springproofs}.
    \begin{theorem}[Springproofs Theorem 5]
        Suppose IPA$_k$ is a~\gls{hvzk} IPA which reduces a relation $R_{zk,k}$ into a relation $R_{zk,k/2}$, and the blinding factors in the two relations distribute independently.
        Given a scheme function $f$, if the SIPA$_{\text{IPA}}(f)$ is terminative for any lengths $n$ of the witness vector, and there exists a polynomial $\texttt{poly}(\lambda)$ such that the number of rounds $m<\texttt{poly}(\lambda)$, then SIPA$_{\text{IPA}}(f)$ is~\gls{hvzk} when $n\geq2$.
    \end{theorem}
    Given this Theorem, we interpret $\text{IPA}_k$ as the Curdleproofs DL~\gls{ipa}.

    In Theorem 5.3.1 of the Curdleproofs paper, they prove their~\gls{ipa} to be zero-knowledge~\cite{Curdleproofs}.
    They do this by help of a simulator and show that the prover's and simulator's response are distributed identically.
    This matches the definition of~\gls{hvzk} from Definition 2, hence the Curdleproofs~\gls{ipa} is~\gls{hvzk}.

    We also know that the~\gls{ipa} is a folding argument, which reduces the size of the argument by half after each iteration.
    In this reduction, Curdleproofs also proved in Theorem 5.3.1 that the values $B_C,B_D,L_{C,j},L_{D,j},R_{C,j},R_{D,j}$ are blinded and identically distributed.

    The scheme function used in CAAUrdleproofs, as seen in~\autoref{fig:fold}(b), is shown by Springproofs to be a variant of their pre-compression method~\cite{zhang2024springproofs}.
    Springproofs show this function to be optimal in the number of folding steps, hence it must also terminate.
    Specifically, the pre-compression is shown to run in $\lceil \log n\rceil$ folding rounds, satisfying the existence of the polynomial mentioned in Theorem 5.

    Curdleproofs show their argument to be zero-knowledge in the random oracle model provided $|\mathbf{G}|\geq8$~\cite{Curdleproofs}.
    Therefore, following Theorem 1, CAAUrdleproofs must be~\gls{hvzk} when $n\geq8$

    \paragraph*{\textbf{Proof of knowledge-soundness and completeness}}
    For soundness and completeness, we refer to Theorem 3 of Springproofs~\cite{zhang2024springproofs}.
    \begin{theorem}[Springproofs Theorem 3]
        Given a terminative SIPA$(f)$, if the number of compression steps in SIPA$(f)$ is $\mathcal{O}(\log n)$, then SIPA$(f)$ is a complete and computational knowledge sound argument of relation (1).
        Moreover, the Fiat-Shamir transformation of SIPA$(f)$ is a non-interactive random oracle argument having completeness and computational knowledge soundness as well.
    \end{theorem}
    Here, relation (1) is
    \begin{align}
        \{(\mathbf{g},\mathbf{h}\in\mathbb{G}^n,u,P\in\mathbb{G};\mathbf{a},\mathbf{b}\in\mathbb{F}_p^n):P=\mathbf{g}^\mathbf{a}\mathbf{h}^\mathbf{b}u^{\langle \mathbf{a},\mathbf{b}\rangle}\}
    \end{align}
    or analogously for an additive cryptographic group:
    \begin{align}
        \{(&\mathbf{g},\mathbf{h}\in\mathbb{G}^n,u,P\in\mathbb{G};\mathbf{a},\mathbf{b}\in\mathbb{F}_p^n):\\
        &P=\mathbf{a}\times\mathbf{g}+\mathbf{b}\times\mathbf{h}+\langle \mathbf{a},\mathbf{b}\rangle u\}\label{al:P}
    \end{align}
    Relating that to Curdleproofs, they mention that they discuss the~\gls{ipa} for the relation
    \begin{align}
        \left\{
        \,(C,D,z;\,\mathbf{c},\mathbf{d})\;\middle|\;
        \begin{aligned}
            C &= \mathbf{c} \times \mathbf{G},\\
            D &= \mathbf{d} \times \mathbf{G'},\\
            z &= \mathbf{c} \times \mathbf{d}
        \end{aligned}
        \right\}
    \end{align}
    Though, taking inspiration from Bulletproofs, which also happens to be the~\gls{ipa} used in Springproofs, they include a commitment to the inner product, $z$, in commitment $C$~\cite{bunz2018bulletproofs}.
    So, before the addition of blinding values and challenges, the relation they want to prove is:
    \begin{align}
        \left\{
        \, \left(
        \begin{aligned}
            \mathbf{G},\mathbf{G'}\in\mathbb{G}^n,\\
            C,D\in\mathbb{G},\\
            z\in\mathbb{F}_p
        \end{aligned}
        \;\middle|\;
        \begin{aligned}
            \mathbf{c},\mathbf{d}\in\mathbb{F}^n_p
        \end{aligned}
        \right)\;\middle|\;
        \begin{aligned}
            C &= \mathbf{c} \times \mathbf{G}+zH,\\
            D &= \mathbf{d} \times \mathbf{G'},\\
            z &= \mathbf{c} \times \mathbf{d}
        \end{aligned}
        \right\}
    \end{align}
    We can now take a look at Springproofs' $P$ commitment in relation to Curdleproofs' $C$ and $D$ commitments.
    If we add together Curdleproofs' two commitment, we get:
    \begin{align}
        C+D=\mathbf{c} \times \mathbf{G}+\mathbf{d} \times \mathbf{G'}+zH
    \end{align}
    This exactly the same commitment as in~\autoref{al:P}.

    Therefore, using Curdleproofs' DL~\gls{ipa} and the pre-compression scheme function, we can instantiate SIPA$(f)$, equivalent to CAAUrdleproofs, as a terminative SIPA$(f)$, with $\mathcal{O}(\log n)$ compression steps.
    Hence, SIPA$(f)$ is a complete and computational knowledge sound argument of relation (1).
    We have just shown that Curdleproofs'~\gls{ipa} proves the same relation, so the properties hold for our SIPA$(f)$ as well.
    Furthermore, Curdleproofs uses the Fiat-Shamir transformation for its verifier challenges.
    So, the SIPA$(f)$, analogously CAAUrdleproofs, is a non-interactive random oracle argument having completeness and computational knowledge soundness as well.

    From this, we can conclude that CAAUrdleproofs is a zero-knowledge argument of knowledge when shuffle size $|\ell|\geq8$.
\end{proof}



