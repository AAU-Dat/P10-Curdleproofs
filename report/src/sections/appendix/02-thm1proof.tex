
\section{Proof of Theorem 1}\label{sec:appendix-thm1proof}
\begin{proof}
    CAAUrdleproofs is the Curdleproofs DL~\gls{ipa} on which the Springproofs protocol has been applied.
    So to help show that it is~\gls{hvzk}, we refer to theorem 5 of Springproofs~\cite{zhang2024springproofs}.
    \begin{theorem}[Springproofs Theorem 5]
        Suppose IPA$_k$ is a~\gls{hvzk} IPA which reduces a relation $R_{zk,k}$ into a relation $R_{zk,k/2}$, and the blinding factors in the two relations distribute independently.
        Given a scheme function $f$, if the SIPA$_{\text{IPA}}(f)$ is terminative for any lengths $n$ of the witness vector, and there exists a polynomial $\texttt{poly}(\lambda)$ such that the number of rounds $m<\texttt{poly}(\lambda)$, then SIPA$_{\text{IPA}}(f)$ is~\gls{hvzk} when $n\geq2$.
    \end{theorem}
    Given this theorem, we interpret$\text{IPA}_k$ as the Curdleproofs DL~\gls{ipa}.

    In theorem 5.3.1 of the Curdleproofs paper, they prove their~\gls{ipa} to be zero-knowledge~\cite{Curdleproofs}.
    They do this by help of a simulator and show that the provers and simulators responses are distributed identically.
    This matches the definition of~\gls{hvzk} from definition 2.
    We also know that the~\gls{ipa} is a folding argument, which reduces the size of the argument by half after each iteration.
\end{proof}


