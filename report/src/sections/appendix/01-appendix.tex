
\section{Definitions of Zero-Knowledge Argument of Knowledge}\label{sec:appendix}
The following definitions are the exact ones also used by Springproofs~\cite{zhang2024springproofs}.
\begin{definition}[Honest-Verifier Zero-Knowledge]
    An interactive argument $(Setup, P, V)$ for a relation $R$ is honest-verifier zero-knowledge, if there exists a
    probabilistic polynomial time simulator, such that there exists a negligible function $\epsilon(\lambda)$, for all adversaries $A_1$ and $A_2$

    \begin{align*}
        \left| % Opening absolute value
        \begin{aligned} % Inner environment for the multi-line content within abs value
            & \Pr \left[ % Alignment point before Pr
                \begin{array}{c|l}
                (\mathbb{x},\mathbb{w}) \in \mathcal{R} \land & \sigma \leftarrow \text{Setup}(1^\lambda) \\
                \mathcal{A}_1(tr) = 1 & (\mathbb{x},\mathbb{w},\rho) \leftarrow \mathcal{A}_2(\sigma) \\
                & tr \leftarrow \text{SIM}(x,\rho)
                \end{array}
                \right] - \\ % End of first line inside aligned
            % Start of second line inside aligned
            & \Pr \left[ % Minus sign, then alignment point before Pr
                \begin{array}{c|l}
                (\mathbb{x},\mathbb{w}) \in \mathcal{R} \land & \sigma \leftarrow \text{Setup}(1^\lambda) \\
                \mathcal{A}_1(tr) = 1 & (\mathbb{x},\mathbb{w},\rho) \leftarrow \mathcal{A}_2(\sigma) \\
                & tr \leftarrow \langle \mathcal{P}(\sigma,\mathbb{x},\mathbb{w}), \mathcal{V}_\rho(\sigma,\mathbb{x}) \rangle
                \end{array}
                \right]
        \end{aligned}
        \right| % Closing absolute value
        \leq \epsilon(\lambda)
    \end{align*}
\end{definition}

\begin{definition}[Non-Interactive Knowledge-Soundness]
    A non-interactive random oracle argument $(Setup, P, V)$ for relation $R$ is knowledge sound, if there exists an efficient knowledge extractor $\mathcal{E}$ and a positive polynomial $z$, such that for any statement $\mathbb{x}\in\{0,1\}^\lambda$ and prover $P^*$ with at most $Q$ queries to the random oracle \texttt{RO},


    \begin{align*}
        \Pr \left[ (\mathbb{x},\mathbb{w}') \in \mathcal{R} \;\middle|\;
        \begin{array}{@{}l@{}} % Using an array for the assignments to the right of the bar
            \sigma \leftarrow \text{Setup}(1^\lambda) \\
            \mathbb{w}' \leftarrow \mathcal{E}_{P^*}(\mathbb{x})
        \end{array}
        \right]
        \geq \frac{\epsilon(P^*, \mathbb{x}) - \kappa(|\mathbb{x}|, Q)}{z(|\mathbb{x}|)},
    \end{align*}
\end{definition}
where $\epsilon(P^*,\mathbb{x}):= \Pr[\langle P^*_{RO}(\sigma,\mathbb{x}),V_{RO}(\sigma,\mathbb{x}) \rangle=1]$, $\kappa(\lambda,Q)\in[0,1]$ is the \textit{knowledge error} and is negligible.
$\varepsilon$~has a black-box oracle access to the prover $P^*$ and can manipulate the random oracle $RO$ for $A$ arbitrarily.

\begin{definition}[Completeness]
    An argument $(Setup, P, V)$ is complete, if for any statement $\mathbb{x}\in L$ and witness $\mathbb{w}$ such that $(\mathbb{x,w})\in R$, there exists a negligible function $\mu(\lambda)$, such that
    \begin{align*}
        \Pr\left[\langle P(\sigma,\mathbb{x},\mathbb{w}), V(\sigma,\mathbb{x})\rangle=1|\sigma\gets Setup(1^\lambda)\right]\geq 1-\mu(\lambda)
    \end{align*}
\end{definition}

