
\section{Curdleproofs Weighted Inner Product Argument Modification Attempt}\label{app:curdleproofs-weighted-inner-product-argument-modification-attempt}
In the following, we will show our unsuccessful attempt in trying to convert the~\gls{dlipa} into a~\gls{wipa}.

Before this, it should be mentioned that we have made code for the~\gls{ipa} in the Curdleproofs repository, which works with Bulletproofs+'
Weighted Inner Product Argument.
Both the prover and verifier work.
The problem is connecting it to the rest of the Curdleproofs protocol.

First, the~\gls{dlipa} stems from the grand product argument that it has been converted from.
Hence, it makes sense to modify the structure of this.

The grand product argument compiles the grand product $p$ down to an equation that can be expressed as an inner product~\cite{Curdleproofs}.
The equation is:
\begin{align}
    p\beta^{\ell}-1=\sum_{i=1}^{\ell}c_i(\beta^i  b_i-\beta^{i-1})
\end{align}
This can be seen as the inner product $z=\mathbf{c}\times\mathbf{d}$, where $z=p\beta^\ell-1$ and $d_i=(\beta^i b_i-\beta^{i-1})$,~$i\in[1,\ell]$.

We will now follow the exact conversion from grand product to inner product as in Curdleproofs.
This includes four steps; \textit{separate}, \textit{compress}, \textit{rearrange}, and \textit{compile}.
But instead of converting from the grand product $p=\Pi_{i=1}^\ell b_i$, we try the conversion with $p=\Pi_{i=1}^\ell b_i^{y_i}$.
Here, $y$ is a Vandermonde vector of challenges, e.g., $y=\{y^0,y^1,\dots,y^{\ell-1}\}$

\subsection{Separate}\label{subsec:separate}
The grand product $p=\Pi_{i=1}^\ell b_i^{y_i}$ has $\ell-1$ multiplication, which we separate into $\ell+1$ checks
\begin{align}
    c_1=1^{y_0}\land c_{i+1}=b_{i}^{y_{i}}c_i,\text{ }i\in[1,\ell)\land p=b_{\ell}^{y_{\ell}}c_{\ell}
\end{align}
As explained by Curdleproofs, the final check $p=b_{\ell}^{y_{\ell}}c_{\ell}$ enforces the grand product $p=\Pi_{i=1}^\ell b_i^{y_i}$.
\subsection{Compress}\label{subsec:compress}
All equations are combined into a single polynomial to ensure that they hold
\begin{align}
    0&=(1-c_1)+(b_{1}^{y_{1}}c_1-c_2)X+(b_{2}^{y_{2}}c_2-c_3)X^2+\\
    &\dots+((b_{\ell-1}^{y_{\ell-1}}c_{\ell-1}-c_\ell))X^{\ell-1}+(b_{\ell}^{y_{\ell}}c_{\ell}-p)X^{\ell}
\end{align}

\subsection{Rearrange \& Compile}\label{subsec:rearrange}
Now, this next step is where the equations start to create problems.
In Curdleproofs, they rearrange the $\mathbf{c}$ terms.
For example, if the shuffle size was 3, the equation would become:
\begin{align}
    c_1(Xb_1-1)+c_2(X^2b_2-X)+c_3(X^3b_3-X^2)
\end{align}
Or equivalently, by using the values of each equation in $\mathbf{c}$:
\begin{align}
    1(Xb_1-1)+b_1(X^2b_2-X)+(b_1b_2)(X^3b_3-X^2)
\end{align}
This can also be stated as:
\begin{align}
    \sum_{i=1}^{\ell}c_i(X^{i}b_{i}-X^{i-1})
\end{align}
Simplifying this equation, it becomes:
\begin{align}
    b_{1}b_{2}b_{3}X^3-1=pX^{\ell}-1
\end{align}
By the Schwartz-Zippel Lemma, as explained by Curdleproofs, the following inner product holds with overwhelming probability if at a random point $\beta$:
\begin{align}
    p\beta^{\ell}-1=\sum_{i=1}^{\ell}c_i(\beta^{i}b_i-\beta^{i-1})
\end{align}
Or just $z=\mathbf{c}\times\mathbf{d}$.

Now, we will do the same thing with our equations, which include the weights $\mathbf{y}$.
Keep in mind that for the~\gls{wipa} to work, we need the following structure:
\begin{align}\label{eq:structure}
    \sum_{i=1}^{\ell}c_i\cdot d_i\cdot y_i
\end{align}
Hence, in the following conversion, that structure is our goal.

Following the compression from the previous section, we will verify whether it retains the structure of~\autoref{eq:structure} after rearrangement and compilation.
Again, we use size 3 for the example.
As in the Curdleproofs case, we rearrange the $\mathbf{c}$ terms to be:
\begin{align}
    c_1(X b_1^{y_1}-1)+c_2(X^2 b_2^{y_2}-X)+c_3(X^3 b_3^{y_3}-X^2)
\end{align}
Again, we insert the values of $\mathbf{c}$:
\begin{align}
    1(X b_1^{y_1}-1)+b_1^{y_1}(X^2 b_2^{y_2}-X)+(b_1^{y_1}b_2^{y_2})(X^3 b_3^{y_3}-X^2)
\end{align}
This simplifies down to:
\begin{align}
    b_1^{y_1}b_2^{y_2}b_3^{y_3}X^3-1=pX^{\ell}-1
\end{align}

Unfortunately, this does not align with the structure we set as our goal in~\autoref{eq:structure}.

Instead, we can work directly from the given structure shown in~\autoref{eq:structure} and see what we get.
We now use $p=\Pi_{i=1}^\ell b_i$:
\begin{align}
    \sum_{i=1}^{\ell}c_i(\beta^{i}b_i-\beta^{i-1})y_i
\end{align}
Here, a problem arises.
If we still look at the example with shuffle size 3:
\begin{align}
    1(Xb_1-1)y_1+b_1(X^2b_2-X)y_2+(b_1b_2)(X^3b_3-X^2)y_3
\end{align}
We are not able to simplify this equation.
As a result of this, the verifier would need to know the values of $\mathbf{c}$ and $\mathbf{d}$ to compute~$z$, which breaks~\gls{zk}.
Hence, the conversion is not useful.

Therefore, significant protocol changes are needed to implement the~\gls{wipa} in Curdleproofs.