\section{Future work}\label{sec:future-works}
In this section, we will focus on areas where the Whisk protocol still has room for improvement.

The main modification from Curdleproofs to CAAUrdleproofs is the added flexibility in choosing the shuffle size for Whisk.
Hence, a topic for future improvements could be proof structure modifications.
The goal of this is to improve the protocol in all cases, including those where the shuffle size is a power of two, for which Curdleproofs and CAAUrdleproofs yield similar results.
As shown in Appendix~\ref{app:curdleproofs-weighted-inner-product-argument-modification-attempt}, we attempted to achieve this using~\glspl{wipa} instead of~\glspl{ipa}.
However, there was not enough time to follow through, as it seemed that significant structural changes were needed for this change to be possible.


Besides trying to make the proof faster and reduce the block overhead, there are also calls for making the protocol more secure.
Specifically, work has already begun trying to make Curdleproofs post-quantum secure~\cite{pqwhisk}.
In this work, they make use of the isogeny-based protocol~\gls{csidh}~\cite{10.1007/978-3-030-03332-3_15}.
Isogeny-based cryptography is based on maps between elliptic curves.
Using isogenies, a hard problem arises, namely the~\gls{gaip}.
\begin{definition}[Group Action Inverse Problem (GAIP)]
    Given a curve $E$, with $End(E)=O$, find an ideal $a\subset O$ such that $E=[a]E_0$
\end{definition}
This problem bears some resemblance to the discrete logarithm problem.
Hence, using this problem, an almost one-to-one conversion using post-quantum cryptography can be done on Whisk, as shown by Sanso~\cite{pqwhisk}.
Currently, however, there does not exist a~\gls{nizk} proof of shuffle based on isogenies.


When using Whisk in the Ethereum blockchain, a list of upcoming proposers is still chosen and published some time before they are needed for duty.
However, because upcoming proposers are published as trackers that can be opened and proven by the chosen validator, attacks such as~\gls{dos} attacks are significantly harder to perform accurately.
Though, the first part of the proposer~\gls{dos} attack involves de-anonymizing validators, as demonstrated by Heimbach et al.~and confirmed by our research~\cite{heimbach2024deanonymizingethereumvalidatorsp2p,ouroldpaper}.
Even if the blockchain is using Whisk, it is still possible for an adversary to gather and de-anonymize validator IP addresses only by running a node on the network.
A sustainable solution for this, therefore, needs to be found.
However, Ethereum is a system that encourages transparency, so a possible solution should take this into account.
