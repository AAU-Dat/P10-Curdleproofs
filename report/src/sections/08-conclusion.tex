

\section{Conclusion}\label{sec:conclusion}
After examining the~\gls{zk}~\gls{ssle} protocol, Whisk, and the Curdleproofs protocol, we found that there was still room for improvement in the Curdleproofs protocol.
We identified the strict requirement of the shuffle size being a power of two as a limitation, and we aimed to remove this limitation to reduce the block overhead related to the protocol.

To achieve this, we drew inspiration from Springproofs, which allows \glspl{ipa} to be of any size.
By combining the Curdleproofs protocol with the flexibility of Springproofs, we made the CAAUrdleproofs protocol.
The implementation of the CAAUrdleproofs protocol is a modified version of the Curdleproofs protocol that allows for any shuffle size.

Through our experiments, we found that the CAAUrdleproofs protocol has similar proving and verifying times to the Curdleproofs protocol when the shuffle size is a power of two.
However, for any shuffle size that is not a power of two, the CAAUrdleproofs protocol has a performance advantage.
An advantage that increases the more below a power of two the shuffle size is.

Since CAAUrdleproofs enables the use of any shuffle size, it can be used to reduce the block overhead related to the protocol without compromising the security of the protocol.

We have shown the security through an experiment inspired by~\cite{cryptoeprint:2022/560}.
Here, we found that the shuffle size can be reduced to 80 and remain secure, also considering the domain in which the protocol is intended to operate.
Using this, we see a block size overhead of 72.33\% compared to that of Curdleproofs.
 

Hence, we have shown CAAUrdleproofs to be an optimized modification of Curdleproofs, as it allows for more flexibility in the choice of shuffle size. 
The optimization is based on reducing the size of the block overhead and achieving faster proving and verifying times.
