

\section{Introduction}\label{sec:introduction}
The Ethereum network is a decentralized network and is one of the most popular blockchains.
At the current time Ethereum is the most used blockchain which relies on the consensus mechanism of~\gls{pos}.
Ethereum switched from~\gls{pow} to~\gls{pos} in 2022 with the release of Ethereum 2.0~\cite{EthereumProof-of-stakePoS}.
A part of this decision was to reduce the requirements for participating in the network.
With the increasing number of validators, the network is becoming more and more active with the amount of messages being sent between the validators.


With all of these messages being sent, the network is becoming increasingly vulnerable to attacks based on information gathering.
Making collecting information about things like IPs of validators possible and creating a potential threat to the network.
This vulnerability is what lead to the paper by Heimbach et al.~\cite{heimbach2024deanonymizingethereumvalidatorsp2p} where they present a de-anonymization attack on validators in the Ethereum network.
That is why we in this paper have recreated the de-anonymization of validators in the Ethereum network from~\gls{de-anon paper} by identifying and linking IPs to active validators.


Even without having a validator, we gather information about the validators in the~\gls{p2p} network through attestations sent in throughout the Ethereum network.
We were able to gather enough information purely from logging the messages we received and the peers we were connected to, to be able to de-anonymize validators.
The information gathered from our attack can then be used to perform attacks such as a~\gls{dos} attack on other proposers when combined with the fact that the proposers of each block is known in advance.
This could potentially lead to a loss of money for the validators, as they are penalized for being inactive as well as a potential gain for the attacker in the event of victim being the proposer in the slot before the attacker.

This paper was inspired by the work of~\cite{heimbach2024deanonymizingethereumvalidatorsp2p} and we have made our own version of the implementation.
This has lead os to make the following contributions:
\begin{itemize}
    \item We have implemented a de-anonymization attack on validators designed to be running on the Ethereum network.
    The attack was run on the Holesky testnet for research purposes.
    \item We have described a possible~\gls{dos} on block proposers, which is made possible after execution of the de-anonymizing validator attack.
    \item We have documented our results of the de-anonymization attack and compared those against the results from the original authors of the attack~\cite{heimbach2024deanonymizingethereumvalidatorsp2p}
    \item We discuss the incentives and consequences that arise after a successful execution of the~\gls{dos} attack on block proposers.
\end{itemize}

\subsection{related work}\label{subsec:related-work}
The usage of~\glspl{zkp} in Ethereum is not a new concept.
In fact, it currently uses them both on- and off-chain.
The following provides a short overview of some of the already existing solutions as well as one still being in development.


\subsubsection{DoS Attack}\label{subsubsec:dos-attack}
Some of the instances of~\gls{dos} attacks that are seen on ethereum ranges from attacks on the proposers to attacks that seek to slow down the network itself.
A known attack aims to slow down the network by using underpriced opcodes to create a block that is hard to process~\cite{10.1145/3391195,9815256}.
Another way to slow down the network is to create empty accounts that are hard to process~\cite{empty-account-mitigation,empty-account-eip-mitigation}.
This attack, however, is outdated and has been mitigated by making it near impossible to create empty accounts in the network.

