\section{Results}\label{sec:results}

\subsection{Connected peers}\label{subsec:connected-peers}
\begin{figure*}[!ht]
    \centering
    \includegraphics[scale = 0.5]{figures/conPeer2}
    \caption{The number of peers our node was connected to over the time of the experiment.
    Data points are collected every minute and the average amount of peers connected is 148.}
    \label{fig:peersconnected}
\end{figure*}
Throughout the experiment, we kept track of the number of peers our node was connected to.
The number of connected peers is shown in~\autoref{fig:peersconnected}.
Our node was connected to at most 300 peers at once, and the average was 148.
Over the course of the experiment, the number of connected peers fluctuated a lot but would continually rise for the first 12 hours.
Then it would heavily drop and rise again.
This pattern would repeat, but after the first drop it would then drop consistently every 7 hours instead.

\subsection{Deanonymization}\label{subsec:deanonymization}


\begin{table}[]
    \centering
    \caption{Distribution of nodes into the four different categories}
    \begin{tabular}{lll}
        \hline
        & \textbf{Nodes} & \textbf{Distribution} \\ \hline
        \textbf{Deanonymized validators} & 513            & 42.93\%                 \\
        \textbf{64 subnets}              & 71             & 5.94\%                  \\
        \textbf{No validators}           & 0              & 0\%                     \\
        \textbf{Rest}                    & 611            & 51.13\%                 \\ \hline
    \end{tabular}
    \label{tab:distribution}
\end{table}


\begin{table}[]
    \centering
    \caption{Number of validators located by our modified Prysm node. The first column indicates the total number of validators. The second column indicates validators with more than one IP-address mapped to it.}
    \begin{tabular}{lll}
        \hline
        & \textbf{Validators} & \textbf{Non-Unique Validators} \\ \hline
        \textbf{Overall} & 493,551             & 158,436                        \\ \hline
    \end{tabular}
    \label{tab:unique vals}
\end{table}

IPs collected = 1183

\subsection{Validator Distribution}\label{subsec:validator-distribution}
From all the peers we were connected to, we collected the number of validators on each peer.
The cumulative distribution of validators on each peer is shown in~\autoref{fig:validatorsonpeers}.
Here we can see that 57\% of the peers that we connected to have at least one validator.
The amount of validators on each peer ranges from 0 to 87028, with the average being 1003 validators per peer.
We see that the most common amount of validators on a peer is 1 with about 7\% of the found peers having only one validator.
We also saw that having 110 and 400 were the most common amounts of validators on a peer once we get above 20 validators on a peer.
We found a total of 24 peers that had 10,000 or more validators on them.
\begin{figure*}[!ht]
    \centering
    \includegraphics[scale = 0.5]{figures/distval}
    \caption{The cdf showing the distribution of validators on each peer}
    \label{fig:validatorsonpeers}
\end{figure*}