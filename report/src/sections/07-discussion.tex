

\section{Discussion}\label{sec:discussion}
In this section we will discuss the results of the experiments in \autoref{sec:results} and how they relate to the CAAUrdleproofs protocol.
We will also discuss some of the limitations of the CAAUrdleproofs protocol and how it compares to Curdleproofs.


\subsection{CAAUrdleproofs in comparison to Curdleproofs}\label{subsec:CAAUrdleproofs-vs-Curdleproofs}




\subsection{Shuffle Security}\label{subsec:Discution-Shuffle-security}
When looking at the results of the shuffle security experiment in \autoref{fig:shufflesecurity} and \autoref{fig:shufflesecurityviolin}, we can see that when taking into account the standard deviation, the shuffle can still be secure with an~$\ell$ as low as 32 within the 8192 shuffles available.
Even when taking into account the worst case scenario from our experiment, the shuffle will still be secure with an~$\ell$ as low as 42 within the 8192 shuffles available.

We would however not recommend using an~$\ell$ lower than 80, as the worst case scenario uses a little under half the available shuffles, and you would only need one third to get within the standard deviation.
This would stil lead to the size of the block overhead and the speed of the protocol being significantly reduced.



